%%%%%%%%%%%%%%%%%%%%%%%%
%Article stuff
%%%%%%%%%%%%%%%%%%%%%%%%
% This package serves to balance the column lengths on the last page of the document.
% please, insert \balance command in the left column of the last page
\usepackage{balance}
\usepackage{supertabular}
\usepackage{paralist}
\usepackage{booktabs}

%% to enable \thank command
\IEEEoverridecommandlockouts 
%% The usage of the following packages is recommended
%% to insert graphics
\usepackage[dvips]{graphicx}
% to typeset algorithms
\usepackage{algorithm}
\usepackage{algpseudocode}
% to typeset code fragments
\usepackage{listings}
% to make an accent \k be available
\usepackage[T1]{fontenc}
% provides various features to facilitate writing math formulas and to improve the typographical quality of their output.
\usepackage[cmex10]{amsmath}
\interdisplaylinepenalty=2500
% por urls typesetting and breaking
\usepackage{url}
% for vertical merging table cells
\usepackage{multirow}
\usepackage{siunitx}


\usepackage{xcolor}
\usepackage[english]{babel}
\newcommand{\thiscolor}[1]{\colorbox{#1}{\phantom{Here is text}}}

%%%%%%%%%%%%%%%%%%%%%%%%
%Mixed packages
%%%%%%%%%%%%%%%%%%%%%%%%
\usepackage[hidelinks]{hyperref}
\usepackage{graphicx}
\usepackage[textsize=scriptsize]{todonotes}
\usepackage{glossaries}
\usepackage[numbers]{natbib}
\usepackage{subcaption}
\bibliographystyle{plainnat}
\usepackage{amssymb}
\let\labelindent\relax %Needed before enumitem to avoid error with IEEE template
\usepackage{enumitem}
% Encoding %
\usepackage[utf8]{inputenc}
\usepackage{enumitem}
\usepackage{framed}


%%%%%%%%%%%%%%%%%%%%%%%%
%Figures
%%%%%%%%%%%%%%%%%%%%%%%%

\usepackage{tikz}
\usepackage{pgfplots}

% Tables %
\usepackage{tabularx}
\usepackage{xltabular}
\newcolumntype{Y}{>{\centering\arraybackslash}X}
\usepackage{booktabs} % for professional tables
\usepackage{tikz}
\usetikzlibrary{positioning}
\usepackage{longtable}
\usepackage{makecell}
\usetikzlibrary{shapes.geometric, arrows}
\usetikzlibrary{matrix,positioning,arrows.meta,arrows}

\tikzset{my arrow/.style={-latex},
	set@com@col/.style={},set@com@col@aryarg/.style={column #1/.style={set@com@col}},
	set@com@row/.style={},set@com@row@aryarg/.style={row #1/.style={set@com@row}},
	set common column/.style 2 args={set@com@col/.style={#2}, set@com@col@aryarg/.list={#1}},
	set common row/.style 2 args={set@com@row/.style={#2}, set@com@row@aryarg/.list={#1}},
}

%%%%%%%%%%%%%%%%%%%%%%%%
%Defines
%%%%%%%%%%%%%%%%%%%%%%%%

\newtheorem{definition}{\textbf{Definition}}

% Autoref
\def\sectionautorefname{Section}
\def\subsectionautorefname{Section}
\def\subsubsectionautorefname{Section}
\def\figureautorefname{Fig.}
\def\definitionautorefname{Definition}

% Plates
\usepackage{amstext}
\usepackage{tikz}
\usetikzlibrary{arrows,decorations.pathmorphing,fit,positioning}

\newcommand{\dir}{\text{Dirichlet}}
\newcommand{\mult}{\text{Multinomial}}


\usepackage{listings}


%Code listing style named "mystyle"
\lstdefinestyle{mystyle}{
	backgroundcolor=\color{white},   
	commentstyle=\color{olive},
	keywordstyle=\color{purple},
	numberstyle=\tiny\color{gray},
	stringstyle=\color{blue},
	basicstyle=\ttfamily\footnotesize,
	breakatwhitespace=false,         
	breaklines=true,                 
	captionpos=b,                    
	keepspaces=true,                 
	numbers=left,                    
	numbersep=5pt,                  
	showspaces=false,                
	showstringspaces=false,
	showtabs=false,                  
	tabsize=2
}

%"mystyle" code listing set
\lstset{style=mystyle}
