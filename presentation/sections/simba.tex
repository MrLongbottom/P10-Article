\section{Introduction}

\begin{frame}{\insertsection}{}
	\begin{itemize}
		\item 
	\end{itemize}
\end{frame}

\begin{frame}{\insertsection}{What is topic modeling?}
	\begin{itemize}
		\item Imagine going through $139,060$ articles, manually
		\item With the goal of finding all sports article.
	\end{itemize}
\end{frame}

\begin{frame}{\insertsection}{What is topic modeling?}
	\begin{itemize}
		\item Topic modeling
  		\begin{itemize}
			\item Data has arisen from a generative process
			\item Discover main themes within a collection of documents.
		\end{itemize}
	\end{itemize}
\end{frame}

\begin{frame}{\insertsection}{What is topic modeling?}
	\begin{figure}
		\includegraphics[width=\textwidth]{figures/topic_modeling_visual.JPG}
		\let\thefootnote\relax\footnote{\tiny{http://www.cs.columbia.edu/~blei/papers/Blei2012.pdf}}
	\end{figure}
\end{frame}

\begin{frame}{\insertsection}{Topic modeling using news articles}
	\begin{figure}
		\includegraphics[width=\textwidth]{figures/danish_newspapers.jpg}
		\let\thefootnote\relax\footnote{\tiny{https://www.fyidenmark.com/denmark-newspapers.html}}
	\end{figure}
\end{frame}

\section{Dataset}

\begin{frame}{\insertsection}{Nordjyske}
	\begin{figure}
		\includegraphics[width=0.6\textwidth]{figures/nordjyske-medier.png}
		\let\thefootnote\relax\footnote{\tiny{https://addfocus.dk/nordjyske-medier-2/}}
	\end{figure}
	\begin{itemize}
		\item 2017-2019
		\item News articles
		\item Multiple \textbf<2>{metadata} fields
	\end{itemize}
\end{frame}


\begin{frame}{\insertsection}{The three chosen metadata}
	\begin{itemize}
		\item Author
		\item Category
		\item Taxonomy
	\end{itemize}
\end{frame}

\subsection{Author}

\begin{frame}{\insertsubsection}{Metadata description}
	\begin{itemize}
		\item The person who has written the article
		\item $184$ unique authors (after preprocessing)
		\item Example: Peter Tordrup Larsen
	\end{itemize}
\end{frame}


\subsection{Category}

\begin{frame}{\insertsubsection}{Metadata description}
	\begin{itemize}
		\item Newspaper or labels
		\item $34$ unique categories (after preprocessing)
		\item Example: Aalborg-avis or Kultur
	\end{itemize}
\end{frame}


\subsection{Taxonomy}

\begin{frame}{\insertsubsection}{Metadata description}
	\begin{itemize}
		\item A hierarchical structure of topical or geographical information 
		\item $355$ unique taxonomy entries (after preprocessing)
		\item Example: PLACES/Danmark/Nordjylland/Aalborg and TOPICS/Religion/Christianity
	\end{itemize}
\end{frame}

\begin{frame}{\insertsection}{Metadata influence}
	\begin{itemize}
		\item What impact does these metadata have on the topics?
	\end{itemize}
\end{frame}

\section{Models}

\subsection{Latent Dirichlet Allocation}

\begin{frame}{\insertsubsection}{latent Dirichlet allocation}
	\begin{figure}
		\centering
		\resizebox{\textwidth}{!}{%
			\begin{tikzpicture}
				[
				observed/.style={minimum size=15pt,circle,draw=blue!50,fill=blue!20},
				unobserved/.style={minimum size=26pt,circle,draw},
				post/.style={->,>=stealth',semithick},
				]
				
				\node (w-j) [observed] at (0,0) {$W_{d,n}$};
				\node (z-j) [unobserved] at (-1.5,0) {$Z_{d,n}$};
				\node (theta) [unobserved] at (-3,0) {$\theta_d$};
				\node (alpha-hyper) [unobserved, label=above:$\alpha$,left of=theta, node distance=2cm] {};
				\node (beta-hyper) [unobserved] at (2.75,0) {$\beta_k$};
				\node (eta-hyper) [unobserved, label=above:$\eta$, ,right of=beta-hyper, node distance=2cm] {};
				
				\path
				(z-j) edge [post] (w-j)
				(alpha-hyper) edge [post] (theta)
				(theta) edge [post] (z-j)
				(beta-hyper) edge [post] (w-j)
				(eta-hyper) edge [post] (beta-hyper)
				;
				
				\node [draw,fit=(w-j) (theta), inner sep=14pt] (plate-context) {};
				\node [above right] at (plate-context.south west) {$D$};
				\node [draw,fit=(w-j) (z-j), inner sep=10pt] (plate-token) {};
				\node [above right] at (plate-token.south west) {$N$};
				\node [draw,fit=(beta-hyper) (beta-hyper), inner sep=17pt] (plate-context) {};
				\node [above right] at (plate-context.south west) {$K$};
			\end{tikzpicture}
		}
	\end{figure}
	\begin{itemize}
		\item $\alpha$ and $\eta$ are topic hyperparameters
		\item $\theta_d$ and $\beta_k$ are the document-topic and topic-word distribution
		\item $Z_{d,n}$ is the topic assignment 
	\end{itemize}
	\let\thefootnote\relax\footnote{\tiny{All pieces are random variables}}
\end{frame}
