In the main paper, we initialize and describe our problem with a focus on results and analysis.
Within this appendix, we are expanding on many aspects of the aforementioned paper and new extensions to the models.
Following is an overview over each section and what it investigates.

\begin{enumerate}
	\item Metadata labels
	\begin{itemize}
		\item The metadata is shown in tables for all three metadata types, and observations about the metadata labels are described.
	\end{itemize} 
	\item Topic coherence
	\begin{itemize}
		\item We explain the purpose and mathematical ideas behind the evaluation metrics we use in the paper.
	\end{itemize} 
	\item Grid search 
	\begin{itemize}
		\item Our grid search process is described further on how we chose our hyperparameters.
	\end{itemize}
	\item Coloring articles 
	\begin{itemize}
		\item Two more articles are highlighted the same way as in \autoref{sec:discussion} and are analyzed.
	\end{itemize}
	\item Stemming the dataset 
	\begin{itemize}
		\item An experiment using a stemmed dataset is described, and the results are shown.
	\end{itemize}
	\item Pyro model implementation
	\begin{itemize}
		\item We describe the probabilistic programming language Pyro which we explored before choosing to work with Gibbs sampling.
	\end{itemize}
	\item Gibbs sampling
	\begin{itemize}
		\item The code for the Gibbs sampler is explained and investigated. A parallel Gibbs sampling method is also mentioned.
	\end{itemize}
	\item Pachinko implementation
	\begin{itemize}
		\item The implementation of our \gls{pam} model and how it differs from the Gibbs sampling method is explained.
	\end{itemize}
	\item Category and author \gls{pam}
	\begin{itemize}
		\item We go into detail about how we are using author and category metadata in the \gls{pam} and what results were achieved.
	\end{itemize}
	\item The author-category model
	\begin{itemize}
		\item We also combine the author-topic and category-topic models into one model with two topic distributions. This model is also analyzed.
	\end{itemize}
	\item The author-doc and category-doc models
	\begin{itemize}
		\item We create two new models called the author-doc and category-doc model. These models are a combination of the standard \gls{lda} and the author and category metadata.
	\end{itemize}
	\item Applications of our models
	\begin{itemize}
		\item We explore the application possibilities of our project and propose various applications which could be implemented at news sites such as Nordjyske.
	\end{itemize}
\end{enumerate}
