In the main paper, we initialize and describe our problem with a focus on results.
Within this appendix, we are expanding on many aspects of the aforementioned paper and describing new experiments that have been made.

This includes:
\begin{itemize}
	\item Overview
	\begin{itemize}
		\item A brief explanation of the process of training and evaluating our models.
	\end{itemize}
	\item Expanding on the paper
	\begin{itemize}
		\item The metadata is shown in tables for all three metadata types, and observations about the metadata labels are described.
		\item We explain the purpose and mathematical ideas behind the evaluation metrics we use in the paper.
		\item Our grid search process is described further on how we chose our hyperparameters.
		\item More articles are highlighted the same way as in \autoref{sec:discussion} and observations are made.
	\end{itemize}
	\item Code
	\begin{itemize}
		\item An experiment using a stemmed dataset is described, and the results are observed.
		\item We describe the probabilistic programming language Pyro which we explored before choosing to work with Gibbs sampling.
		\item The code for the Gibbs sampler is explained and how the \gls{pam} differs from the standard Gibbs Sampler is described. 
		\item Our testing of a parallel Gibbs sampler is also described.
	\end{itemize}
	\item Expanded model experiments
	\begin{itemize}
		\item We go into detail how we explored using author and category metadata in the \gls{pam} and what results were achieved.
		\item We also explored a model combining the author-topic and category-topic models into one model with two topic distribution. This model is also analyzed.
		\item Further model combinations we explored, were models that combine the standard \gls{lda} with our author and category metadata models.
	\end{itemize}
	\item Applications
	\begin{itemize}
		\item We explore the application possibilities of our project and propose various applications which could be implemented at Nordjyske.
	\end{itemize}
\end{itemize}
