\subsection{Gibbs Sampling}\label{sec:appendix_gibbs}
The Gibbs Sampling algorithm consists of two procedures: Random Initialization and Gibbs Sampling.
In the following sections, we explain how these procedures have been implemented.

\subsubsection{Random Initialization}
Before a Gibbs sampling algorithm can run, word occurrences need to be randomly initialized with topics.
To do this, we iterate over each word within each document and assign a random topic for it.
This procedure is shown in \autoref{lst:randomInit}.
\begin{lstlisting}[language=Python, caption=Random Initialization,label={lst:randomInit}, float, floatplacement=H]
import numpy as np

def rand_initialize(documents: List[np.ndarray]):
	wt_assignment = []
	for doc in documents:
		curr_doc = []
		for word in doc:
			# Construct the topic distribution
			pz = _conditional_distribution()
			
			# Draw a new topic and assign it
			t = np.random.multinomial(1, pz).argmax()
			curr_doc.append(t)
			
			# Increase the topic counts
			increase_count()
		wt_assignment.append(curr_doc)
	return wt_assignment
\end{lstlisting}

Various parameters are left out to simplify the code listing of our initialization method.
The \emph{\_conditional\_distribution()} function creates a distribution of topics based on the current word.
It is based on \autoref{eq:gibbs_eq}~\cite{author_topic_2012}.
\begin{equation}\label{eq:gibbs_eq}
	\begin{split}
		P(z_i = k|w_i = m, \boldsymbol{z}_{-i}, \boldsymbol{w}_{-i}, \alpha, \beta) = 
		\underbrace{\frac{C^{DT}_{dk} + \alpha}{\sum_{k'} C^{DT}_{dk'} + T\alpha}}_{Doc-Topic}
		\underbrace{\frac{C^{WT}_{mk} + \beta}{\sum_{m'} C^{WT}_{m'k} + V\beta}}_{Topic-Word}
	\end{split}
\end{equation}
where $C^{DT}_{dk}$ is the number of times document $d$ uses topic $k$.

$C^{WT}_{mk}$ is the number of times topic $k$ uses word $m$.
$\boldsymbol{z}_{-i}$ represents the topic assignments where the current instance is disregarded.
$\alpha$ and $\beta$ is the Dirichlet parameter for the document-topic and topic-word distribution, respectively.
The first fraction describes how likely topic $t$ appearing in document $d$ is, and the second fraction describes which words are most probable in topic $t$.
Following the code in \autoref{lst:randomInit}, as we initialize, the words get higher probability of clustering together, since we update the counts every time, we look at a new word in line 16.

\subsubsection{Gibbs Sampling}
Continuing the analogy from before, we can start to investigate the Gibbs sampling method itself, where we iterate over each word in every document and draw a new topic based on the given topic distribution.
As in \autoref{lst:randomInit}, the code has been simplified.
\begin{lstlisting}[language=Python, caption=Gibbs Sampling Method,label={lst:gibbsSampling}, float, floatplacement=H]
import numpy as np

def gibbs_sampling(documents: List[np.ndarray],
	   doc_topic_dist: np.ndarray,
	   doc_topic_count: np.ndarray,
	   topic_word_dist: np.ndarray,
	   topic_word_count: np.ndarray,
	   wt_assignment: List[List[int]]):

	for d_index, doc in documents:
		for w_index, word in enumerate(doc):
			# Get the topic of the current word
			topic = wt_assignment[d_index][w_index]
			
			# Decrease the topic count
			decrease_count()
			
			# Sample a new topic
			pz = _conditional_distribution()
			topic = np.random.multinomial(1, pz).argmax()
			
			# Assign topic to the current word
			wt_assignment[d_index][w_index] = topic
			
			# And increase the topic count
			increase_count()
\end{lstlisting}
The Gibbs sampling method is very similar to the random initialization method in \autoref{lst:randomInit}, but with a few extra additions. 
Now we introduce the \emph{decrease\_count()} which decreases the topic count for both words and documents, as the \emph{increase\_count()} increases them.
The sampling explained in \autoref{lst:randomInit} is the same.
On line 13, we get the current topic for the given word and on line 23 we assign a newly drawn topic to that word.
