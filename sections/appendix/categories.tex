\subsection{Metadata labels}\label{sec:appendix_meta_data}

\subsubsection{Category}\label{subsec:appendix_category}

Here are the $58$ different categories, which make up all the category labels present within the dataset.
Categories with fewer than 139 documents ($0.1$ of the number of documents) are removed as part of preprocessing and replaced with a single miscellaneous category 'misc'.
This reduces the number of categories to $34$, while only replacing 292 documents to have the 'misc' category.
This preprocessing also makes the size of the remaining categories much more evenly distributed, as can be seen in \autoref{fig:category_hist}.

\begin{figure*}[ht]
	\centering
	\begin{subfigure}{0.45\textwidth}
		\centering
		\includegraphics[width=\linewidth]{figures/category_hist2_before.png}
		\caption{Before preprocessing}
		\label{fig:category_hist_before}
	\end{subfigure}
	\begin{subfigure}{0.45\textwidth}
		\centering
		\includegraphics[width=\linewidth]{figures/category_hist2_140.png}
		\caption{After preprocessing}
		\label{fig:category_hist_after}
	\end{subfigure}
	\caption{Histogram over the number of categories for different number of documents, before and after preprocessing.
	Categories on the x-axis are grouped into 50 buckets.}
	\label{fig:category_hist}
\end{figure*}

As mentioned in \autoref{sec:dataset}, some nonsensical categories are worth mentioning.
Specifically, the categories '26. Frederik' and '53. Frederik' do not from their names indicate what they cover, since Frederik is simply a normal Danish name.
Since they are not filtered into the 'misc' category during preprocessing, it is worth looking into what documents have these categories.

By looking at a random selection of documents from these categories, there is no clear pattern to be found.
The topics, within these categories, can be about anything from sports to news from anywhere around the world or locally in Denmark.
The articles from '26. Frederik' appear all years in our dataset, while the articles for '53. Frederik' seem to be just from 2019, the last year in our dataset, but evenly distributed over the whole year.
Curiously enough, most of the documents from these categories seem to be written by Anders Kjærgaard, with only a few other authors.
For '26. Frederik' there are just 5 unique authors: Carsten Tolbøll, Emil Abkjær Kristensen, Morten Kyndby Holm, Jens Fogh-Andersen, and Anders Kjærgaard.
For '53. Frederik' there are just 4 unique authors: Klaus Færch Gjerulff, Morten Kyndby Holm, Jens Fogh-Andersen, and Anders Kjærgaard.
This shows that Morten Kyndby Holm, Jens Fogh-Andersen, and Anders Kjærgaard have written for both categories.

From the exploration of these two categories, we can not say with certainty why these categories exist, or why multiple authors have chosen to write for these categories.
We continue to use these categories in our experiment, for the possibility of making other observations through the topic models.

\begin{table*}[h]
	\centering
	\begin{tabular}{l|c|l|c|l|c|l|c}
		Category            & Number & Category        & Number & Category                       & Number & Category                    & Number \\
		\toprule
		Fælles              & 20204  & Navne           &  3749  & Kram                           &  244   & \textbf{Østvendsyssel Avis} &   4    \\
		Thisted-avis        & 11473  & Kultur          &  3012  & 53. Frederik                   &  203   & \textbf{DF Motor Biler}     &   3    \\
		Sport-avis          & 10941  & Morsø Sport     &  2350  & Feature                        &  188   & \textbf{Nyhedsmotoren-net}  &   3    \\
		Debat               & 10075  & Friii           &  2333  & \textbf{Aalborg:nu}            &   73   & \textbf{Plus Publicering}   &   3    \\
		Udland-avis         &  8855  & Bagside         &  1933  & \textbf{Erhvervsnavne}         &   39   & \textbf{RB}                 &   3    \\
		Erhverv-avis        &  7356  & MitLiv          &  1519  & \textbf{Newspack}              &   35   & \textbf{Sport-net}          &   3    \\
		Mariagerfjord-avis  &  7241  & WEEKEND         &  1493  & \textbf{DF Søfart}             &   32   & \textbf{Thisted-net}        &   3    \\
		Morsø-avis          &  5959  & Bo Godt         &  1447  & \textbf{Morsø Ugeavis}         &   27   & \textbf{Hanbo-bladet}       &   2    \\
		Aalborg-avis        &  5544  & Nordjyske Biler &  1400  & \textbf{DF Licitation Byggeri} &   14   & \textbf{Brugermappe}        &   1    \\
		Vesthimmerland-avis &  5131  & Morsø Debat     &  1375  & \textbf{Biler}                 &   13   & \textbf{Brønderslev-net}    &   1    \\
		Rebild-avis         &  4415  & Frieord         &  1341  & \textbf{Samfund}               &   9    & \textbf{Lokalavisen}        &   1    \\
		Frederikshavn-avis  &  4325  & Indsigt         &  984   & \textbf{Nordjyske Plus}        &   6    & \textbf{Mariagerfjord-net}  &   1    \\
		Hjørring-avis       &  4235  & Thisted sport   &  698   & \textbf{Oplandsavisen}         &   6    & \textbf{Morsø-net}          &   1    \\
		Brønderslev-avis    &  3857  & Perspektiv      &  613   & \textbf{INFOMAKER PRINT}       &   5    &                             &        \\
		Jammerbugt-avis     &  3791  & 26. Frederik    &  484   & \textbf{DF Licitation Diverse} &   4    &                             &        \\
		\bottomrule
	\end{tabular}
	\caption{Amount of documents for each of the 58 categories within the Nordjyske dataset from 2017 to 2019.
		The highlighted categories are filtered and combined during preprocessing.}
	\label{tab:category_table}
\end{table*}
\todo[inline]{mark categories that are preprocessed away}


\subsubsection{Author}\label{subsec:appendix_author}
Unlike with Categories, the Author metadata field does not have a natural cut-off point, with a good amount of values in a specific lower range, followed by more evenly distributed numbers.
Instead, the vast majority of the values fall in a lower range, meaning that setting too high a cut-off point, will result in removing a large portion of the data.
We instead choose a lower cut-off point, keeping most of the authors, except the ones that contained so few documents, that finding common topics would be inefficient.
Authors, who have written less than $14$ documents ($0.01\%$ of the number of documents), are removed as part of preprocessing.
This removes $43$ out of $227$ authors, combining them into a single 'misc' author.
A total of $204$ documents are assigned to the 'misc' author.
\todo[inline]{figure out whether to use boxplots or histograms, and ref them.}

\begin{figure*}[ht]
	\centering
	\begin{subfigure}{0.45\textwidth}
		\centering
		\includegraphics[width=\linewidth]{figures/author_hist2_before.png}
		\caption{Before preprocessing}
		\label{fig:author_hist_before}
	\end{subfigure}
	\begin{subfigure}{0.45\textwidth}
		\centering
		\includegraphics[width=\linewidth]{figures/author_hist2_14.png}
		\caption{After preprocessing}
		\label{fig:auhtor_hist_after}
	\end{subfigure}
	\caption{Histogram over amount authors who have written certain number of documents, before and after preprocessing.
	Authors on the x-axis are grouped into 50 buckets.}
	\label{fig:author_hist}
\end{figure*}

\begin{figure*}[ht]
	\centering
	\begin{subfigure}{0.45\textwidth}
		\centering
		\includegraphics[width=\linewidth]{figures/author_box_before.png}
		\caption{Before preprocessing}
		\label{fig:author_box_before}
	\end{subfigure}
	\begin{subfigure}{0.45\textwidth}
		\centering
		\includegraphics[width=\linewidth]{figures/author_box_14.png}
		\caption{After preprocessing}
		\label{fig:auhtor_box_after}
	\end{subfigure}
	\caption{Boxplot over the amount of documents written by authors.}
	\label{fig:author_box}
\end{figure*}

\subsubsection{Taxonomy}\label{subsec:appendix_taxonomy}
Taxonomy is fundamentally different from the other metadata fields, in this paper.
It is not fully observed with only roughly $25\%$ of documents having a taxonomy field.
It is hierarchical, with each taxonomy containing a sequence of sub-taxonomies, such as: 'STEDER/Danmark/Nordjylland/Aalborg'.
It is also possible for each document to have multiple taxonomy sequences.
We remove any sub-taxonomy that is used in less than $4$ taxonomy sequences.
Out of $1135$ sub-taxonomies, $522$ are removed during this preprocessing.
\todo[inline]{describe layer sizes of taxonomy tree}

\begin{table}
	\begin{tabular}{l | c | c | c | c | c}
		Metadata & Min & Max & Mean & Median & Std. \\
		\hline
		Author & 1 & 9893 & 612.6 & 191 & 1219.7 \\
		Author (preprocessed) & 15 & 9893 & 751.7 & 323 & 1311.9 \\
		Category & 1 & 20204 & 2397.6 & 548 & 3812.1 \\
		Category (preprocessed) & 188 & 20204 & 4090.0 & 2681 & 4227.3 \\
		Taxonomy & 1 & 29535 & 123.0 & 6 & 1385.9 \\
		Taxonomy (preprocessed) & 5 & 29535 & 227.6 & 17 & 1879.2 \\
	\end{tabular}
	\caption{Statistics over documents associated with metadata values, before and after preprocessing}
	\label{tab:meta_prepro_stats}
\end{table}
