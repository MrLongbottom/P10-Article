\subsection{Perplexity}\label{app:perplexity}
One of the evaluation metric, that is commonly used to evaluate topic models, is perplexity.
Perplexity is used as a metric, to show how well a model can predict new test samples $w_d$.
As described in \autoref{sec:related_work}, \citet{tea_leaves} states that perplexity does not infer semantically meaningful topics.
Because of this, we have chosen not to use perplexity in our experiments.
But because perplexity is not specific to topic models, and by itself does not give an indication of how coherent topics are, it is mainly used as a secondary evaluation~\cite{tea_leaves}.
To calculate perplexity, we first need to compute the log-likelihood of $w_d$, which is done in:
\begin{equation}\label{eq:likelihood}
	\mathcal{L}(w_d) = \log p(w_d|\Phi) = \sum_{d} \log p(w_d|\Phi)
\end{equation}
\noindent where $\Phi$ is the topic-word matrix.
The perplexity measure is then calculated as follows:
\begin{equation}
	\emph{Perplexity}(w_d) = exp \{-\frac{\mathcal{L}(w_d)}{W}\}
\end{equation}
\noindent where W is the number of words \cite{de2008evaluating}.

%\begin{table}[h]
%	\centering
%	\caption{Results.}
%	\begin{tabular}{l|c}
%		Topic Model & Perplexity \\
%		\midrule
%		\Acrlong{lda} & 6761.6 \\
%		Author-Topic Model & 10991.5 \\
%		Category-Topic Model & 11126.0 \\
%		Taxonomy-Topic Model & - \\
%		Author-Category Model & 64260.6  \\
%		Author-Doc Model & 888036.4  \\
%		Category-Doc Model & 896045.7 \\
%	\end{tabular}
%	\label{tab:app_perplexity}
%\end{table}