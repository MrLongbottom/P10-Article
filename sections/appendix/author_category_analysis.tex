\subsubsection*{Author-category model analysis}
While the results of the models with a single metadata are the most interesting, looking at the results of the Author-Category model may also bring new observations.
In \autoref{tab:_extra_metric_results} the metric results for this model is shown, and in \autoref{tab:all_gibbs_topic_examples}, random samples of topics from the \gls{lda}, author-topic, category-topic, and author-category model can be seen.
For the author-category model, the top words in the topics do for the most part not make a lot of sense, though, the author-topic and category-topic models also have topics that are difficult to understand.
It seems, to some degree, that the topics are a mix of the top words of the two combined metadata topic distributions, which makes sense since we draw word topics from a multiplication of these.
While the topics may be less understandable, having a topic distribution for authors and categories gives more opportunities for, e.g., applying these in article recommendation, compared to only having one topic distribution.