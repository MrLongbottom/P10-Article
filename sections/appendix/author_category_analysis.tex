\subsubsection*{Author-category model analysis}
While modeling single metadata fields is the main focus of this paper, looking at the results of models combining multiple metadata may also bring new observations.
For this purpose, we will be examining the results of the Author-Category model.
In \autoref{tab:_extra_metric_results} the metric results for this model are shown, and in \autoref{tab:all_gibbs_topic_examples}, random samples of topics from the \gls{lda}, author-topic, category-topic, and author-category model can be seen.
For the author-category model, the top words in the topics are mostly semantically incoherent, although the author-topic and category-topic models also have topics that are difficult to understand.
It seems, to some degree, that the topics are a mix of the top words of the two combined metadata topic distributions, which makes sense since we draw word topics from the multiplication of these.
While the topics may be less understandable, having a topic distribution for authors and categories gives more opportunities for, e.g., applying these in article recommendation, compared to only having one topic distribution.
