\subsection{Category and author \gls{pam}}\label{app:cat_auth_pachinko}
Given the good results of the taxonomy-topic model, we decided to test \gls{pam} with the two other metadata types: Author and Category.
These metadata are not layered and can therefore not utilize the layered nature of \gls{pam} in the same way.
Instead, a Four-Level \gls{pam} is used, with a root layer, a metadata layer where authors and categories are locked into topics using the technique outlined in \autoref{subsec:pam}, a layer with 90 'blank' topics, and a word layer.

As can be seen from the results in \autoref{tab:cat_auth_pachinko_results}, these models achieve better results than the previous author and category models and the \gls{lda} model.
However, the taxonomy-topic model is still better overall.
The same conclusion is reached after manual inspection of the topics of these models.

\begin{table}[b]
	\centering
	\caption{Topic coherence of author \gls{pam}, category \gls{pam}, and \gls{pam} without metadata (marked with bold) compared to previous models.}
	\begin{tabular}{l|c}
		Topic Model & \makecell{Topic \\ Coherence} \\
		\midrule
		\Acrlong{lda} & 0.520 \\
		Author-topic model & 0.335 \\
		Category-topic model & 0.370 \\
		Taxonomy-topic model & 0.660 \\
		\textbf{Author \gls{pam}} & \textbf{0.598} \\
		\textbf{Category \gls{pam}} & \textbf{0.585} \\
		\textbf{\Acrlong{pam}} & \textbf{0.670} \\
	\end{tabular}
	\label{tab:cat_auth_pachinko_results}
\end{table}

For comparison we also ran a Four-Level \gls{pam} without any metadata, using 100 and 90 as the sizes of the two middle layers.
This ended up providing very good results, slightly better than the taxonomy-topic model.
The elapsed time of the Four-Level \gls{pam} was ${\sim}128$ hours for $50$ epochs, roughly the same as our Five-Level taxonomy-topic model, which had an elapsed time of $132$ hours for $50$ epochs.
The slower speed compared to the size of the \gls{dag} structure is due to this model being unable to skip observed values when sampling since it does not incorporate any metadata.
This also shows that while Category \gls{pam} and Author \gls{pam} get better results than their \gls{lda} counterparts, it is better to run \gls{pam} without modifications.
This could be due to our models being unable to make good use of the extra information provided by the metadata.
It might also point towards these two particular metadata types not being particularly useful in this specific project.
The category metadata is generally very vague and some categories have seemingly no connection between documents.
As discussed earlier, the author metadata might not be as useful within journalism as authors don't write about the same topics as with scientific papers.
And while the \gls{pam} without metadata does achieve better topic coherence than the taxonomy-topic model, they are close enough in both topic coherence and manual inspection of the quality of topics that no conclusions can be drawn.
