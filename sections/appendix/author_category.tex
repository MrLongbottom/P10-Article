\subsection{The author-category model}\label{sec:combination}
\vejleder[inline]{hvorfor og formål?}
We have created a combination model, as an extension of our metadata models, to see whether using multiple metadata fields at the same time to draw topics, would affect the performance of the topic model.
The idea is that this model combination should give insight into what a model learns when multiple metadata influence the topics chosen.
The model we have created is the Author-Category combination model.
As the name suggests, this model includes an author-topic distribution and a category-topic distribution, and the plate notation can be seen in \autoref{fig:author_category_lda}.

To combine the Author and Category metadata, we multiply the topic distributions $\theta_a$ and $\theta_c$ when sampling a new word topic.
Though, before sampling the word topic, we normalize the product of the multiplication to sum to 1, which also makes it a distribution. \vejleder[inline]{mathify this}
This makes both of the learned meta distributions influence the sampled word topics.

\begin{figure*}[ht]
	\centering
	\resizebox{.3\textwidth}{!}{%
		\begin{tikzpicture}
	[
	observed/.style={minimum size=26pt,circle,draw=blue!50,fill=blue!20},
	unobserved/.style={minimum size=26pt,circle,draw},
	post/.style={->,>=stealth',semithick},
	]
	
	\node (w-j) [observed] at (0,0) {$W_{d,n}$};
	\node (z-j) [unobserved, above of= w-j, node distance=2.5cm] {$Z_{d,n}$};
	\node (author_obs) [observed, above of= z-j, node distance=2.5cm] {${a_d, c_d}$};
	\node (author_dist) [unobserved, left of=z-j, node distance=2.5cm] {$\theta_a$};
	\node (category_dist) [unobserved, left of=author_obs, node distance=2.5cm] {$\theta_c$};
	\node (alpha-hyper) [unobserved, label=above:$\alpha$, above of=eta-hyper, node distance=3.75cm] {};
	\node (beta-hyper) [unobserved, left of = w-j, node distance=2.5cm] {$\beta_k$};
	\node (eta-hyper) [unobserved, label=above:$\eta$, left of=beta-hyper, node distance=2cm] {};
	
	\path
	(z-j) edge [post] (w-j)
	(alpha-hyper) edge [post] (author_dist)
	(alpha-hyper) edge [post] (category_dist)
	(category_dist) edge [post] (z-j)
	(author_obs) edge [post] (z-j)
	(author_dist) edge [post] (z-j)
	(beta-hyper) edge [post] (w-j)
	(eta-hyper) edge [post] (beta-hyper)
	;
	
	\node [draw,fit=(w-j) (author_obs), inner sep=14pt] (plate-context) {};
	\node [below right] at (plate-context.north west) {$D$};
	
	\node [draw,fit=(w-j) (z-j), inner sep=10pt] (plate-token) {};
	\node [below left] at (plate-token.north east) {$N$};
	
	\node [draw,fit=(beta-hyper) (beta-hyper), inner sep=17pt] (plate-context) {};
	\node [above right] at (plate-context.south west) {$K$};
	
	\node [draw,fit=(author_dist) (author_dist), inner sep=17pt] (plate-context) {};
	\node [above right] at (plate-context.south west) {$A$};
	
	\node [draw,fit=(category_dist) (category_dist), inner sep=17pt] (plate-context) {};
	\node [above right] at (plate-context.south west) {$C$};
	
\end{tikzpicture}
	}
	\caption{Plate notation for the Author-Category \gls{lda} \vejleder[inline]{ny model?}model.}
	\label{fig:author_category_lda}
\end{figure*}

\subsection{Author-category model analysis}
While the results of the models with a single metadata are the most interesting\vejleder[inline]{i hvilken sammenhæng?}, looking at the results of the Author-Category model may also bring new observations.
In \autoref{tab:_extra_metric_results} the metric results for this model is shown, and in \autoref{tab:all_gibbs_topic_examples}, random samples of topics from the \gls{lda}, author-topic, category-topic, and author-category model can be seen.
For the author-category model, the top words in the topics do for the most part not make a lot of sense\vejleder[inline]{uddyb}, though, the author-topic and category-topic models also have topics that are difficult to understand.
It seems, to some degree, that the topics are a mix of the top words of the two combined metadata topic distributions, which makes sense since we draw word topics from a multiplication of these.
While the topics may be less understandable, having a topic distribution for authors and categories gives more opportunities for, e.g., applying these in article recommendation, compared to only having one topic distribution.

\begin{table}
	\caption{A sample of random topics' top 10 words, for the \gls{lda}, author-topic, category-topic, and author-category model. 
	Each section in this table presents 15 random topics, where each topic is randomly picked from the model on the left and each line represents a topic.}
	\label{tab:all_gibbs_topic_examples}
	\centering
	\begin{tabular}{c|c}
		Model & Topics \\
		\midrule
		\multirow{15}{*}{\gls{lda}} & millioner, direktør, sæby, seneste, tv, mener, stadig, landbrug, fokus, bedre \\
		& stjalet, indbrud, klokken, nordjyllands, thisted, politi, politiet, mandag, villa, oplyser \\
		& omradet, boliger, natur, naturen, ligger, du, vand, dyr, a, skov \\
		& du, hans, ham, arige, mig, sagde, stedet, folk, liv, min \\
		& millioner, procent, tv, selskabet, milliarder, skriver, største, aars, direktør, københavn \\
		& thy, thisted, mors, unge, nielsen, arbejde, jensen, a, frederikshavn, folk \\
		& omradet, kommunen, boliger, kr, projektet, byen, a, millioner, by, nyt \\
		& arige, politiet, mand, arig, politi, retten, sagen, fængsel, ham, oplyser \\
		& børn, du, børnene, hvordan, min, mor, forældre, livet, mennesker, skole \\
		& biler, km, bilen, kr, hk, bil, t, a, vw, motor \\
		& hobro, hadsund, mariager, morgen, mariagerfjord, bio, the, sker, kl, filmteatret \\
		& mig, min, hans, liv, altid, du, mennesker, maske, verden, lille \\
		& millioner, bank, sagen, nordjyske, penge, dansk, sag, skat, lars, sagde \\
		& ebh, bank, finn, finansiel, kunst, lørdag, indbrud, vendsyssel, nordjylland, banks \\
		& km, kr, hk, t, bilen, thisted, bil, biler, a, m \\
		\midrule
		\multirow{15}{*}{Author-topic} & du, formand, tale, fem, kr, betyder, dermed, mal, arets, ligger \\
		& jens, ford, vif, london, januar, team, problemerne, eh, vendsyssel, bla \\
		& seneste, bjarne, gruppen, vendsyssel, erik, abent, middelboe, lavendel, nationalpark, motorvejen \\
		& foie, karstensen, elin, bonderup, fredrik, derhjemme, hector, hee, kjøller, lillian \\
		& skriver, hobro, sine, kommuner, dk, jammerbugt, set, min, mig, bedste \\
		& jobi, tilværelse, crowdfunding, klippekortet, knude, nyby, thea, bpa, regi, judo \\
		& du, sine, formand, seneste, jensen, hvert, nyt, hvordan, finde, kommunen \\
		& guaido, fordele, albert, smed, forslag, fie, tørken, krævede, egon, tingene \\
		& rundt, netop, gange, mig, gik, kr, større, landet, universitet, livet \\
		& du, dansk, thisted, mig, procent, eu, ned, arbejde, hans, mener \\
		& mig, millioner, skriver, ham, kommunen, hver, nordjylland, unge, sine, mand \\
		& set, glas, odense, vesthimmerland, leth, markedet, trump, ni, regionerne, prins \\
		& bælum, juul, udlændingenævnet, fruevej, vaarst, svitlana, jernstøberi, bislev, bannere, lo \\
		& carl, resultat, poul, krabbe, p, ansat, begynde, holger, ledelse, g \\
		& ned, procent, arige, eu, made, ham, først, større, mennesker, lyder \\
		\midrule
		\multirow{15}{*}{Category-topic} & foregar, passer, imidlertid, yde, mængder, parlamentet, boris, henvendelser, white, berg \\
		& yderste, sæsonen, lykkes, jernbaner, salgsprisen, efterladt, kakeeto, aab, frygt, rigtigt \\
		& du, børn, mig, hans, unge, procent, mener, politiet, hvordan, thisted \\
		& mener, langt, seneste, ting, mors, give, egen, hurtigt, seks, nej \\
		& du, thisted, dansk, unge, mig, børn, a, hans, procent, arbejde \\
		& jasmin, chelsea, rahbek, norden, partnerskabet, malstregen, eva, modernisering, festligt, byggefirmaet \\
		& nordjyske, hjørring, giver, forhold, hobro, heller, mors, rundt, række, mulighed \\
		& etiske, træningstilbud, lei, statuen, raab, torsdagscafe, aula, pattedyr, berømmelsen, ydet \\
		& omtumlet, sydvendt, gla, dine, golde, trilogi, guidning, jungersen, areal, konservatorer \\
		& nordjyske, sagde, plads, made, dette, fredag, omradet, heller, fald, byen \\
		& sine, hver, skabe, juni, lars, tyskland, vendsyssel, michael, interesse, din \\
		& min, jorden, dit, udfordring, thomas, datter, konkurrence, situation, museum, drive \\
		& the, løbet, stjalet, regering, gaet, tredje, sikkert, byens, omradet, turen \\
		& du, min, gode, gamle, ad, henrik, eu, finde, sat, hobro \\
		& bedre, thy, haft, ham, hver, gik, synes, lars, millioner, eksempel \\
		\midrule
		\multirow{15}{*}{Author-category} & ebh, klare, glas, kvaliteten, finansiel, rebecca, karrieren, storgaard, tørre, børnehave \\
		& du, unge, sagde, samtidig, procent, bedste, hans, brønderslev, hvordan, kommende \\
		& socialdemokratiske, placeres, ellemann, laustsen, fischer, san, regnskabsar, ordentligt, vejgaard, symptomer \\
		& grønland, norden, morris, kenneth, logan, taliban, niki, robinson, tonnies, quinoa \\
		& ganske, største, rød, tyrkiet, tilfældet, dybt, bo, f, fodbold, utrolig \\
		& tror, klar, disse, handler, a, pedersen, holder, mors, borgmester, november \\
		& lola, børnetallet, anklagemyndighed, lagring, madbar, daimler, fortænke, celtic, videnskabsfolk, lokalitet \\
		& reducerede, forældrepar, utrygt, opgøres, pmi, judy, fusk, claude, matine, forsikrings \\
		& egebjerg, overbelægning, gundersen, floden, karrenbauer, involvere, imerco, udviklingsomrader, hh, skygger \\
		& drive, elektrificeres, bryggeriforeningen, fortjenstmedalje, brændstofpriser, sports, afmærket, ball, fanebærer, legal \\
		& velfærdsstat, fortæl, nævneværdigt, solskin, adsbøl, børnesoldater, uvelkomne, reaktioner, daniel, messing \\
		& benn, hadet, knortegas, højskolens, højreradikale, vietnamesere, sner, florence, partiprogram, puk \\
		& du, børn, hans, unge, thisted, hvordan, arige, procent, mand, sagde \\
		& frederiksen, skabt, tæt, halv, sociale, jammerbugt, jørgensen, norge, danskere, ligesom \\
		& skarpere, venskab, landvind, china, motionsform, spørges, drøner, tværfaglige, islændinge, bevæget \\
	\end{tabular}
\end{table}
