\subsection{Pyro model implementation}
%XTried different approaches, one being probabilistic programming languages. Pyro
%XBasics behind Pyro (uses SVI for instance)
%Implemented a basic version first
%Expanded their ProdLDA example with metadata information
%Switched to Gibbs sampling to have opportunities to change things in detail
After we decided to work with metadata in \gls{lda} models, we had to firstly get a standard model implemented.
Here we decided to look in a few directions for the best way to implement a generative model, and we found that the probabilistic programming language Pyro could be used.
Pyro is a probabilistic programming language that is written in Python with PyTorch as a backend.
This makes it ideal for making quick implementations of models, with the possibility of using tensors and sampling with PyTorch distribution methods.
Pyro also has a built-in stochastic variational inference class that simplifies the training of a model.
These features made it an ideal programming language to look into.

