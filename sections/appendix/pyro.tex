\subsection{Pyro model implementation}
At the beginning of the project, we decided to look in a few directions for the best way to implement an \gls{lda} model, and we found that the probabilistic programming language Pyro could be used~\cite{bingham2019pyro}.
Pyro is a probabilistic programming language that is written in Python with PyTorch as a backend.
This makes it ideal for making quick implementations of models, with the possibility of using tensors and sampling with PyTorch distribution methods.
Pyro also has a built-in stochastic variational inference class that simplifies the training of a model.
These features made it an ideal programming language to look into.

We found that Pyro has made two code examples of \gls{lda} available, one using purely distributions and the other using a neural network approach.
We tested both of these examples to see if we could adapt an existing implementation to work with our dataset.
Pyro's first example of an \gls{lda} model is very simple\footnote{Amortized Latent Dirichlet Allocation: \url{https://pyro.ai/examples/lda.html}}.
It is a functional model, but it has the limitation that all documents need to have the same number of words.
This limitation is too restrictive for most datasets, including ours, and changing the model to handle any number of words makes the model unable to converge.

The other model example from Pyro is called ProdLDA\footnote{ProdLDA: \url{https://pyro.ai/examples/prodlda.html}}.
This model uses an autoencoding approach where the model encodes the document-topic distribution $\theta$ and decodes the topic-word distribution $\beta$.
This approach can handle more complex models and seems to be able to handle the inclusion of metadata.
While this model is able to learn with our dataset, it restricts the possibility of changing the model since the main distributions of the model are learned with a neural network.
These two examples show that Pyro can be useful when making basic topic models, but since we want to extend \gls{lda} with metadata, Pyro becomes too restrictive.
To give us more opportunities to customize our models and not be restricted by Pyro's implementation, we chose to implement the models from scratch with Gibbs sampling.
