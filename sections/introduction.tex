\section{Introduction}\label{sec:introduction}
Datasets consist of entries where each entry contains a certain amount of meta-data fields which describes the data.
Datasets are sometimes not used to their full potential, regarding the amount of information available within some datasets.
There exist many feature engineering method, which capture different aspects of a dataset, and these might aid in solving a specific task.
Meta-data is describes field within a data entry which and is usually used with supervised learning as labels for the algorithm used.
These meta-data fields within a data entry can be used for many purposes eg. to predict new samples within a certain scenario.

Within the field of topic modelling, \Gls{lda} proven successful in finding underlying topics within a corpus of text documents.
Many extension of the \gls{lda} algorithm have been proposed to improve the original model.
One of the well known extensions is the Author-Topic model by \citet{author_topic} which combines the \gls{lda} model which models the relationship between authors and the documents, which they have written.
The reason for modelling this relationship is to figure out which subjects certain authors write about and how much each author use specific words. 
They also show that the Author-Topic combination yields better and more coherent topics, which begs the question whether any other metadata can be applied in a similar fashion.
What if the underlying topics, within a corpus could found with the help of features already present in the data.

These topics are based on the document that resides in the corpus and words within these documents, but other extensions of \gls{lda} have included various other information such as authors, into the model, to generate better topics.
In this paper, we will research the effect that including various combinations of metadata information about documents will have on \gls{lda}.
These metadata include authors as well as higher level categorical information.
We will conduct \gls{lda} models using various combinations of these metadata and observe the effect of the topic generated.
We use a dataset of danish newspaper articles, and construct models that account for the specific metadata available for this dataset.