\section{Introduction}\label{sec:introduction}
% LDA / Topic Modeling Intro
Within the field of topic modeling, \gls{lda} has proven successful in finding underlying topics within a corpus of text documents.
These topics are constructed based on the documents in the corpus and words within these documents. 
However, other extensions of \gls{lda} have also modeled various other information to generate better topics and/or topics with different focuses and potential uses.

% Author-Topic
One of the well-known extensions is the Author-Topic model by \citet{author_topic_2012} that combines the \gls{lda} model with author information to model the relationship between authors and the documents, which they have written.
This is based on the assumption that most authors usually write about only a few different topics.
By modeling this relationship, they find connections between authors and topics, and between different authors, giving more underlying information to the topics.
They also show that the Author-Topic combination yields better and more coherent topics, which begs the question of whether any other document-related data can be applied similarly.

% Topic Modeling dataset
In \gls{lda}, the data used is a corpus of documents, with each document containing a sequence of words.
However, the dataset that will form this corpus also often includes various other data than just the text within the documents, such as authors, time of publishing, tags, etc.
% Metadata definition
We define this other data as metadata, that is data related to the documents in the corpus, but not directly a part of the text.
In most models, metadata is not used.
However, there is some previous work that attempts to make full use of the dataset available by incorporating various metadata into topic models.

% In this paper
In this paper, we will be taking a closer look at a specific dataset and construct different topic models that incorporate various metadata.
We do this in order to see if they can improve the quality of the topics or the efficiency of the model.
We will also investigate the differences between the produced topics for various models, and evaluate their potential uses.

We will be using a dataset from Nordjyske, a danish news agency that maintains multiple local newspapers throughout the North Jutland region.
We will construct topic models that account for the specific metadata available within this dataset.
This metadata includes author information, as well as higher-level categorical information, which will be described in \autoref{sec:dataset}.

% Problem Statement
With these areas of focus, we can define the problem we investigate:

\begin{itemize}
	\item \textit{How does including metadata within the \gls{lda} model impact the resulting topics?}
	\item \textit{How can multiple metadata fields be integrated within a topic model?}
\end{itemize}
\todo[inline]{contributions}
\todo[inline]{Possibilities for adding more to the section: write more about our specific approach, still on overview/abstract level. Write a little after the problem statement about what the questions cover/will give.}

% Paper Structure
The paper is organized as follows: 
\todo[inline]{describe paper structure}
