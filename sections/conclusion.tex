\section{Conclusion}\label{sec:conclusion}
We explore possibilities of incorporating meta-data into existing topic models, such as \gls{lda} and \gls{pam}.
We evaluate three different metadata, Author, Category, and Taxonomy, each of which represent the data in a different way.
From the topic coherence results, shown in \autoref{tab:metric_results}, the \gls{pam} model using the Taxonomy meta data gets the best results.
The Author and Category models are the worst performing, where the topic coherence is much lower than the other models.
We want to answer the problem statement, which we stated in \autoref{sec:introduction}:

\begin{itemize}
	\item \textit{How does including metadata within the \gls{lda} model impact the resulting topics?}
	\item \textit{What possible problems can these models help alleviate at Nordjyske?}
\end{itemize}

Incorporating metadata into a topic model can be done various ways, and we incorporate them in multiple ways.
The \gls{lda} model is used as a baseline, which indicates whether incorporating metadata can improve the topic quality of our topic models.
Based on our analysis, we see that only using the metadata for topic assignment within \gls{lda}, can hurt the topic quality, which is how Author-Topic and Category-Topic is run.

We use the \gls{pam} to incorporate a hierarchical structured metadata called Taxonomy, where we use a novel locking mechanism to lock the observed metadata's topic into place.
Using this technique and model, we are able to get better topic quality and words within \gls{pam} compared to the \gls{lda}.
However, the run time for the algorithm is quite slower than the \gls{lda}.




