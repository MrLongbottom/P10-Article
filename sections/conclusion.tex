\section{Conclusion}\label{sec:conclusion}
In this paper, we explore possibilities for incorporating metadata\vejleder{gerne ??} into existing topic models, such as \gls{lda} and \gls{pam}.
We evaluate these models based on three different metadata: Author, Category, and Taxonomy, each of which represents the data in a different way.

From the topic coherence results, shown in \autoref{tab:metric_results}, the \gls{pam} model using the Taxonomy metadata gets the best results.
The Author and Category models based on \gls{lda} are the worst-performing, where the topic coherence is much lower than the other models.\vejleder{uddyb gerne denne paragraf}

We want to answer the problem statement, which we stated in \autoref{sec:introduction}:
\vejleder{lidt rodet måske.}

\begin{itemize}
	\item \textit{How does including metadata within the \gls{lda} model impact the resulting topics?}
	\item \textit{What possible problems can these models help alleviate at Nordjyske?}
	\begin{itemize}
		\item \textit{More specifically, how can these models improve recommendation?}
	\end{itemize}
\end{itemize}

We incorporate metadata into our topic models in multiple ways.
The \gls{lda} model is used as a baseline, which indicates whether incorporating metadata can improve the topic quality of our topic models.
Based on the results of Author-Topic and Category-Topic, we see that only using the metadata for topic assignment within \gls{lda} can hurt the topic quality.
Other studies have shown that only using the metadata for topic assignment can improve the performance.
This might be due to the particularities of the dataset, such as authors not usually writing about the same subject within the news environment.  


We use the \gls{pam} to incorporate a hierarchically structured metadata called Taxonomy, where we use a novel locking mechanism to lock the observed metadata's topics into place.
Using this model and technique, we can get better topic quality and words within \gls{pam} compared to the \gls{lda}.
However, the run time of the algorithm is quite slow compared to the \gls{lda}.
All models ran for 50 epochs, but \gls{pam} had a runtime of 130 hours, compared to \gls{lda} which had a runtime of 8 hours.

Topic modeling can be used to alleviate many problems at Nordjyske, such as recommendations.
We can use the topic distributions from our models to compare articles based on their similarity in topics.
For example, the author-topic model's topic distribution can be used to recommend similar authors, while in the topic distributions of the taxonomy model, there is the possibility of looking at topics from different layers.
This information can be used in a content-based filtering approach to recommend similar articles.
We suggest using this topic information to enrich the recommendation process at Nordjyske. 
