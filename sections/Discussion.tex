\section{Discussion}\label{sec:discussion}
In this section, we investigate the different models to see how each meta-information field impacts the resulting topics.
Firstly, we want to investigate the top words within each model.
We have taken a random article from the dataset and visualized how the topics differ between the models. 
Before investigating the article below, we define a specific color scheme for each model.

In the article, we have highlighted the highest probable words within the three most occurring topics in the article.
The article is about agriculture and how farmers are letting their cows out onto grass fields in September. 
It also mentions a few different farms in the Northern part of Jutland and describes these in various ways.

The way we can compare these models, is by taking the top 200 words for each topic-word distribution within each model and marking them in the article below.
Since the Author and Category models do not have a document-topic distribution we can not look at the specific document, but instead we have marked the words from the given category- and author-topic distribution within the document, to see what the difference in topics are.
We are looking at the top 3 topics within each model for the specific document.

\begin{table}[h]
	\centering
	\caption{Color scheme for each model.}
	\begin{tabular}{l|c}
		Topic Model & Color \\
		\midrule
		\Acrlong{lda} & \thiscolor{Goldenrod} \vspace*{2mm} \\
		Author-Topic Model & \thiscolor{Aquamarine} \vspace*{2mm} \\
		Category-Topic Model & \thiscolor{LimeGreen} \vspace*{2mm} \\
		Word appearing in multiple models & \thiscolor{Peach} \vspace*{2mm}  \\
	\end{tabular}
	\label{tab:disc_color}
\end{table}

\emph{
Kig på grise, køer og kyllinger10 \colorbox{Peach}{nordjyske} bedrifter åbner \colorbox{LimeGreen}{søndag} for stalddørene Landbruget åbner \colorbox{LimeGreen}{søndag} 16. \colorbox{Peach}{september} ladeporte og stalddøre for offentligheden. 52 danske bedrifter er med i årets ”Åbent landbrug”. I det \colorbox{Peach}{nordjyske} kan man kigge forbi på 10 \colorbox{Peach}{forskellige} landbrug. Blandt de \colorbox{Peach}{nordjyske} deltagere er der \colorbox{Peach}{mulighed} for at få indsigt i både kvæg- og svinebedrifter, \colorbox{Peach}{ligesom} en producent af slagtekyllinger \colorbox{Aquamarine}{byder} velkommen. Sidstnævnte kan opleves hos Rokkedahl i Farstrup. De er tre familier med i alt \colorbox{Peach}{seks} børn, der sammen driver Rokkedahl Landbrug med slagtekyllinger og planteproduktion \colorbox{Peach}{samt} Rokkedahl Energi, som laver energioptimering. Herudover har de \colorbox{Goldenrod}{eget} slagteri, hvor ca. 35 af deres i alt 65 \colorbox{Peach}{medarbejdere} arbejder. Familien Rokkedahl har \colorbox{Peach}{arbejdet} med kyllinger siden 1963 og er tredje generation. I staldene og i de omkringliggende folde har de både fritgående og økologiske slagtekyllinger. Velfærdskyllingerne går i flokke og har adgang til store folde. På årsbasis opdrætter Rokkedahl \colorbox{Peach}{otte} \colorbox{Peach}{millioner} kyllinger som enten slagtes på deres \colorbox{Goldenrod}{eget} slagteri eller sælges til eksterne slagterier. På de 1350 \colorbox{Goldenrod}{hektar} har de hvede, byg raps, havre, rug, ærter og hestebønner. Det anvendes primært til foder til velfærdskyllingerne. De dyrker \colorbox{Peach}{jorden} primært økologisk og anvender halmen til opvarmning af staldene. De har varmevekslere på alle stalde for at minimere varmeforbruget og ammoniakudledningen til omgivelserne. Britt og Klaus Kristiansen på Solbakken Agri ved Aabybro er \colorbox{Peach}{klar} til vise en stor, \colorbox{Peach}{dansk} mælkeproduktion frem. Familien tæller også de fire børn, Maria på 18 år, Daniel på 16 år, Kamilla og Laura på 13 år, og de er sjette generation på gården, som de overtog i 2013. Solbakken har 600 økologiske malkekøer, som tilsammen \colorbox{Peach}{giver} 17.000 liter mælk om dagen. Den bliver hentet og kørt til et af Arlas mejerier, hvor den bliver anvendt til økologiske mejeriprodukter. 575 \colorbox{Goldenrod}{hektar} \colorbox{Peach}{land} tilhører gården, og her producerer \colorbox{Goldenrod}{familien} foder til deres \colorbox{Peach}{dyr} \colorbox{Peach}{samt} andre fødevarer.  I Himmerland kan man besøge Sanne og Ole Mathiasen, der driver Nørregaard på Braulstrupvej 9 i Suldrup. Her kan man se søer, smågrise og slagtesvin i staldene og \colorbox{Aquamarine}{høre} om \colorbox{Goldenrod}{produktion} af velfærdsgrise, se maskinerne, få smagsprøver fra Danish Crown og på \colorbox{Peach}{lokale} fødevarer, og \colorbox{Aquamarine}{høre} om biavl. For \colorbox{Aquamarine}{børnene} er der \colorbox{LimeGreen}{leg} i korncontainer og halm, pedaltraktorbane og ponytrækketure. Der er kaffe og kagebord. Åbent \colorbox{Goldenrod}{landbrug} foregår \colorbox{LimeGreen}{søndag} fra \colorbox{LimeGreen}{klokken} 10 til 16. Det er gratis at deltage. Sidste år deltog 96.000 \colorbox{LimeGreen}{danskere} i åbent landbrug.
}

Overall, we see that there is a large amount of overlap between the models, which is interesting since the models use different meta-information to create the various topic distributions.
This indicates that the models share many of the top words, while also indicating a slight deviation between the models due to the meta-information.
The \gls{lda} model shows words like "landbrug" (agriculture) and "produktion" (production), which is what the article is mostly about.
This behavior is to be expected since the performance of \gls{lda} has been explored and evaluated before. 
Author-topic specific words are not very present and are only showing three unique words: "byder", "hører", and "børnene".
This indicates that the Author-topic model has trouble generalizing what the author of this article (Peter Tordrup Larsen) is writing about, possibly because he has written $5002$ articles in our dataset.
Another aspect of the Author-Topic model is that the authors writing these articles most likely do not write about just one subject, which explains why there is only three less important words marked here. 
The Category-topic model only shows four unique words: "søndag", "leg", "klokken", and "danskere".
These words are also very abstract and can be used in many different scenarios.

An interesting part of this analysis is the words appearing in multiple models.
Some of the words within this category are: "arbejdet", "jorden", "dyr", and "nordjyske".
These words summarize the text pretty well, but it is also hard to summarize this text since it covers a wide variety of specific topics.
Combining the results of these models might yield better topic models, but we can not conclude that based on only one article.
There is also the possibility that choosing another random article would give completely different numbers of marked words per model, because this highly depends on the article's author and category.


\subsection{Author-topic model}\label{sec:discussion_author_topic}
Some interesting observations can also be made specifically in the author-topic model.
One observation that is possible, is looking at the similarity of authors.
In this model, the author-topic distribution defines the probabilities of topics being written by a specific author.
Then, just as \citet{author_topic_2012}, the similarity of authors can be found by calculating the symmetric Kullback-Leibler divergence:

\begin{equation}
	sKL(i,j) = \sum_{t=1}^{T}\left[\theta_{it}\, log \frac{\theta_{it}}{\theta_{jt}} + \theta_{jt}\, log \frac{\theta_{jt}}{\theta_{it}}\right]
\end{equation}
\noindent where $\theta_{it}$ is the probability of author $i$ having written about topic $t$, and the same for $\theta_{jt}$ with author $j$.

In the context of using these similarities to assist Nordjyske, knowing how similar authors are gives the opportunity to recommend new authors to readers, while the articles are about similar topics.
In \autoref{tab:author_similarity} the top 10 author pairs, based on this similarity measure, are shown.
A smaller KL value means the authors are more similar.
The number in parenthesis next to each author is the number of articles they have written in our dataset.

\begin{table}[h]
	\centering
	\caption{Top 10 author pairs based on the symmetric KL divergence between authors.}
	\begin{tabular}{r|c}
		Author pair & KL \\
		\midrule
		Lars Termansen (328) \& Mikkel Færgemann Viken (91) & 1.50 \\
		Morten Nis Klenø (17) \& Anne Helene Thomsen (606) & 1.72 \\
		Lars Termansen (328) \& Lars Christensen (1293) & 2.43 \\
		Esben Heine Pedersen (1689) \& Caspar Birk (71) & 2.47 \\
		Lars Christensen (1293) \& Poul Christoffersen (65) & 2.53 \\
		Lone Beck (92) \& Max Melgaard (587) & 2.74 \\
		HANNE Lindblad Jensen (27) \& Peter Tordrup Larsen (5002) & 2.94 \\
		Søren Kjær (95) \& Carl Chr. Madsen (785) & 2.98 \\
		Heidi Majgaard B. Pedersen (244) \& Lisbeth Helleskov (361) & 3.05 \\
		Lars Termansen (328) \& Morten Lind (413) & 3.16 \\
		\midrule
		Maximum & 34.51 \\
		Median & 24.20 \\
	\end{tabular}
	\label{tab:author_similarity}
\end{table}

In general for these pairs, there does not seem to be a correlation between a high similarity and the categories of the articles they have written.
While one author in a pair might have written for the sports category (Sport-avis) the other author might not have written for this category at all.
This can also be seen for categories that cover geographic locations, where one author might have written for Aalborg (Aalborg-avis) and the other author can have written for Thisted (Thisted-avis).

When looking at a sample of documents for the most similar author pair (Lars Termansen \& Mikkel Færgemann Viken), it is seen that they both write a mix of regular news and sports articles.
The reason why they become this similar, might be that the ratio between news and sports news for both authors is similar, and possibly also because of the types of news they write about.
Another interesting observation is that, for the second most similar author pair (Morten Nis Klenø \& Anne Helene Thomsen) the difference in the number of articles written is significant.
Here Morten Nis Klenø has written just 17 articles while Anne Helene Thomsen has written 606 articles.
This suggests that some part of why these authors' similarity is high, simply dependents on the types of news the authors have written, no matter the amount.

It is also worth noting that while authors that write scientific papers usually write in just a few subject areas, the scientific area they work in, this is not the case for news article authors.
In our dataset, this can be seen in the fact that the authors have written for 7.86 categories on average, with 7 categories as the median.
This can make it more difficult for the author-topic model to find patterns in what the authors write about, especially since each category can cover multiple topics.

A selection of authors from the dataset and the top words from their most probable topics, can be seen in \autoref{tab:author_top_words}.

\begin{table*}[h]
	\centering
	\caption{Selection of authors and the top 10 words from their most probable topic.}
	\begin{tabular}{c|c|c|c|c|c|c}
		\toprule
		Birgitte Bové & Kirsten Østergaard & Pauline Bülow & Karen Marie Foldbjerg & Claus T. Kræmmergård & Hanne Lindblad Jensen & Ole Jensen \\
		\midrule
		Topic 41 & Topic 50 & Topic 3 & Topic 13 & Topic 88 & Topic 2 & Topic 50 \\
		\midrule
		\makecell{millioner \\ eu \\ hans \\ større \\ bedre \\ formand \\ kr \\ nordjyske \\ taget \\ skriver} & \makecell{du \\ thisted \\ unge \\ mig \\ børn \\ procent \\ hans \\ hver \\ penge \\ hjørring} & \makecell{procent \\ bag \\ rigtig \\ lave \\ dansk \\ formand \\ gode \\ klar \\ svært \\ plads} & \makecell{du \\ sine \\ formand \\ seneste \\ jensen \\ hvert \\ nyt \\ hvordan \\ finde \\ kommunen} & \makecell{du \\ procent \\ unge \\ børn \\ arige \\ hans \\ dansk \\ mig \\ thisted \\ mener} & \makecell{du \\ thisted \\ procent \\ mig \\ børn \\ hans \\ unge \\ dansk \\ mener \\ a} & \makecell{du \\ thisted \\ unge \\ mig \\ børn \\ procent \\ hans \\ hver \\ penge \\ hjørring} \\
		\bottomrule
	\end{tabular}
	\label{tab:author_top_words}
\end{table*}

As can be seen through this analysis, this knowledge about authors and their topic probabilities can be useful for making better news recommendation systems, but it will be limited since news authors usually write about multiple subjects.
