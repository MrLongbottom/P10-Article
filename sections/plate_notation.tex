\section{Topic Models}\label{sec:plate_notation}
In this section, topic models that are explored in the experiment are detailed.
This covers the standard \gls{lda} from \citet{blei2003latent}, our category and author metadata models, which build on the concept of \gls{lda}, and our taxonomy metadata model, which uses the \acrlong{pam}.

\subsection{Standard \gls{lda}}
The purpose of \gls{lda}, and topic models in general, is to create a tool for exploring collections of text.
Topic models do this by uncovering the underlying semantic structure of a text collection by using hierarchical Bayesian models.
\Gls{lda} uncovers this semantic structure by discovering patterns of word use in documents and finding topics based on these~\cite{blei2009topic}.

The standard \gls{lda} by \citet{blei2003latent} can be described by the following generative process, which is the way the model assumes the documents arose:
$D$ is the number of documents in the corpus, $N_d$ is the number of words in document $d$, $V$ is the size of the vocabulary, and $K$ is the number of topics.
Topics are represented as distributions over words and documents are represented as distributions of topics.
\Gls{lda} assumes that the topics are shared across the corpus, while the document-topic distributions are unique for each document.
For each topic $k \in \{1,\dots, K\}$ a topic-word distribution $\beta_k$ is sampled from a V-dimensional Dirichlet distribution parameterized by $\eta$.
That is, K topics $\beta_{1:k}$ are sampled, each being a distribution over the vocabulary, written as: $\beta_k \sim Dirichlet(\eta)$.
Likewise, for each document $d \in \{1,\dots, D\}$ a document-topic distribution $\theta_d$ is sampled from a K-dimensional Dirichlet distribution parameterized by $\alpha$.
For each word $n \in \{1, \dots, N_d\}$ in each document $d$, a topic $z_{d,n}$ is sampled from a K-multinomial distribution $\theta_d$, and then a word $w_{d,n}$ is sampled from a V-multinomial distribution $\beta_{z_{d,n}}$.
The generative process for each document is seen in these steps:

\vspace{\topsep}
\begin{enumerate}
	\item Draw topic proportion $\theta_d \sim Dirichlet(\alpha)$
	\item For each word $n$ in the document:
	\begin{enumerate}
		\item Draw topic assignment $z_{d,n} \sim Mult(\theta_d)$
		\item Draw word $w_{d,n} \sim Mult(\beta_{z_{d,n}})$
	\end{enumerate}
\end{enumerate}
\vspace{\topsep}

The generative process for topics and documents, generates a list of $K$ topics and $D$ documents that can be used as a $K \times V$ matrix of topic-word distributions and a $D \times K$ matrix of document-topic distributions, respectively.
The plate notation for \gls{lda} can be seen in \autoref{fig:standard_lda}.

\begin{figure}[h]
  \centering
  \resizebox{\columnwidth}{!}{%
	  \begin{tikzpicture}
	    [
	      observed/.style={minimum size=15pt,circle,draw=blue!50,fill=blue!20},
	      unobserved/.style={minimum size=26pt,circle,draw},
	      post/.style={->,>=stealth',semithick},
	    ]
	
	    \node (w-j) [observed] at (0,0) {$W_{d,n}$};
	    \node (z-j) [unobserved] at (-1.5,0) {$Z_{d,n}$};
	    \node (theta) [unobserved] at (-3,0) {$\theta_d$};
	    \node (alpha-hyper) [unobserved, label=above:$\alpha$,left of=theta, node distance=2cm] {};
	    \node (beta-hyper) [unobserved] at (2.75,0) {$\beta_k$};
	    \node (eta-hyper) [unobserved, label=above:$\eta$, ,right of=beta-hyper, node distance=2cm] {};
	    
	    \path
	    (z-j) edge [post] (w-j)
	    (alpha-hyper) edge [post] (theta)
	    (theta) edge [post] (z-j)
	    (beta-hyper) edge [post] (w-j)
	    (eta-hyper) edge [post] (beta-hyper)
	    ;
	    
	    \node [draw,fit=(w-j) (theta), inner sep=14pt] (plate-context) {};
	    \node [above right] at (plate-context.south west) {$D$};
	    \node [draw,fit=(w-j) (z-j), inner sep=10pt] (plate-token) {};
	    \node [above right] at (plate-token.south west) {$N$};
	    \node [draw,fit=(beta-hyper) (beta-hyper), inner sep=17pt] (plate-context) {};
	    \node [above right] at (plate-context.south west) {$K$};
	  \end{tikzpicture}
  }
	\caption{Plate notation for \gls{lda}.}
	\label{fig:standard_lda}
\end{figure}

After training the \gls{lda} model, there are multiple possibilities for exploring the corpus using the posterior distributions of the hidden random variables.
One possibility is to visualize the posterior topics of the model, e.g., by sorting $\beta_k$ according to the highest probabilities.
It is also possible to visualize the documents by, e.g., sorting by the highest topic proportions in $\theta_d$.
Another possibility of exploration is finding similar documents by using a distribution distance function on the topic proportions $\theta_d$ between documents~\cite{blei2009topic}.

\subsection{Author-Topic \gls{lda} and Category-Topic \gls{lda}}
We model both of the metadata fields 'Author' and 'Category' similarly to the model presented by \citet{author_topic_2012}.
In this model, there are no document-topic distributions $\theta$.
Instead, each author and category has its own topic distribution.
This is based on the assumption that authors prefer to write about specific topics, and that categories of the articles were chosen based on the content of the finished article or that local newspapers have their own unique topic preferences.

Our topic models are \glspl{um}, meaning that topics are only word distributions and are chosen based on the data of the documents, rather than \glspl{dm} where topics are fitted to have an influence on both word distributions and other metadata.

\citet{author_topic_2012} describe the author-topic model where for each document $d$, they assign a vector of authors $a_d$ from a set of authors $A$, and for each word draw an author $x$ from this vector.
However, for our category-topic model and our own author-topic model, each document $d$ is associated with one category $c_d$ from a set of categories $C$ and one author $a_d$ from a set of authors $A$, rather than a vector.
This is due to our dataset never having more than one author or category for each document.
Unless otherwise stated, future mentions of the author-topic model are to our implementation of this model, rather than the model presented by \citet{author_topic_2012}.
The plate notation for our category and author \gls{lda} models can be seen in \autoref{fig:metadata_lda}.

\begin{figure*}[ht]
	\centering
	\begin{subfigure}{0.25\textwidth}
		\centering
		\resizebox{\textwidth}{!}{%
		\begin{tikzpicture}
	[
	observed/.style={minimum size=26pt,circle,draw=blue!50,fill=blue!20},
	unobserved/.style={minimum size=26pt,circle,draw},
	post/.style={->,>=stealth',semithick},
	]
	
	\node (w-j) [observed] at (0,0) {$W_{d,n}$};
	\node (z-j) [unobserved, above of= w-j, node distance=2.5cm] {$Z_{d,n}$};
	\node (category_obs) [observed, above of= z-j, node distance=1.5cm] {$c_d$};
	\node (category_dist) [unobserved, left of=z-j, node distance=2.5cm] {$\theta_c$};
	\node (alpha-hyper) [unobserved, label=above:$\alpha$, left of=category_dist, node distance=2cm] {};
	\node (beta-hyper) [unobserved, left of = w-j, node distance=2.5cm] {$\beta_k$};
	\node (eta-hyper) [unobserved, label=above:$\eta$, left of=beta-hyper, node distance=2cm] {};
	
	\path
	(z-j) edge [post] (w-j)
	(alpha-hyper) edge [post] (category_dist)
	(category_obs) edge [post] (z-j)
	(category_dist) edge [post] (z-j)
	(beta-hyper) edge [post] (w-j)
	(eta-hyper) edge [post] (beta-hyper)
	;
	
	\node [draw,fit=(w-j) (category_obs), inner sep=14pt] (plate-context) {};
	\node [below right] at (plate-context.north west) {$D$};
	
	\node [draw,fit=(w-j) (z-j), inner sep=10pt] (plate-token) {};
	\node [below right] at (plate-token.north west) {$N$};
	
	\node [draw,fit=(beta-hyper) (beta-hyper), inner sep=17pt] (plate-context) {};
	\node [above right] at (plate-context.south west) {$K$};
	
	\node [draw,fit=(category_dist) (category_dist), inner sep=17pt] (plate-context) {};
	\node [above right] at (plate-context.south west) {$C$};
\end{tikzpicture}
		}
		\caption{Category \gls{lda}.}
		\label{fig:category_lda}
	\end{subfigure}
	\hspace{2em}
	\begin{subfigure}{0.25\textwidth}
		\centering
		\resizebox{\textwidth}{!}{%
			\begin{tikzpicture}
	[
	observed/.style={minimum size=26pt,circle,draw=blue!50,fill=blue!20},
	unobserved/.style={minimum size=26pt,circle,draw},
	post/.style={->,>=stealth',semithick},
	]
	
	\node (w-j) [observed] at (0,0) {$W_{d,n}$};
	\node (z-j) [unobserved, above of= w-j, node distance=2.5cm] {$Z_{d,n}$};
	\node (author_obs) [observed, above of= z-j, node distance=1.5cm] {$a_d$};
	\node (author_dist) [unobserved, left of=z-j, node distance=2.5cm] {$\theta_a$};
	\node (alpha-hyper) [unobserved, label=above:$\alpha$, left of=category_dist, node distance=2cm] {};
	\node (beta-hyper) [unobserved, left of = w-j, node distance=2.5cm] {$\beta_k$};
	\node (eta-hyper) [unobserved, label=above:$\eta$, left of=beta-hyper, node distance=2cm] {};
	
	\path
	(z-j) edge [post] (w-j)
	(alpha-hyper) edge [post] (author_dist)
	(author_obs) edge [post] (z-j)
	(author_dist) edge [post] (z-j)
	(beta-hyper) edge [post] (w-j)
	(eta-hyper) edge [post] (beta-hyper)
	;
	
	\node [draw,fit=(w-j) (author_obs), inner sep=14pt] (plate-context) {};
	\node [below right] at (plate-context.north west) {$D$};
	
	\node [draw,fit=(w-j) (z-j), inner sep=10pt] (plate-token) {};
	\node [below right] at (plate-token.north west) {$N$};
	
	\node [draw,fit=(beta-hyper) (beta-hyper), inner sep=17pt] (plate-context) {};
	\node [above right] at (plate-context.south west) {$K$};
	
	\node [draw,fit=(author_dist) (author_dist), inner sep=17pt] (plate-context) {};
	\node [above right] at (plate-context.south west) {$A$};
\end{tikzpicture}
		}
		\caption{Author \gls{lda}.}
		\label{fig:author_lda}
	\end{subfigure}
	\hspace{2em}
	\begin{subfigure}{0.25\textwidth}
		\centering
		\resizebox{\textwidth}{!}{%
			\begin{tikzpicture}
	[
	observed/.style={minimum size=26pt,circle,draw=blue!50,fill=blue!20},
	unobserved/.style={minimum size=26pt,circle,draw},
	post/.style={->,>=stealth',semithick},
	]
	
	\node (w-j) [observed] at (0,0) {$W_{d,n}$};
	\node (z-j) [unobserved, above of= w-j, node distance=2.5cm] {$Z_{d,n}$};
	\node (author_obs) [observed, above of= z-j, node distance=2.5cm] {${a_d, c_d}$};
	\node (author_dist) [unobserved, left of=z-j, node distance=2.5cm] {$\theta_a$};
	\node (category_dist) [unobserved, left of=author_obs, node distance=2.5cm] {$\theta_c$};
	\node (alpha-hyper) [unobserved, label=above:$\alpha$, above of=eta-hyper, node distance=3.75cm] {};
	\node (beta-hyper) [unobserved, left of = w-j, node distance=2.5cm] {$\beta_k$};
	\node (eta-hyper) [unobserved, label=above:$\eta$, left of=beta-hyper, node distance=2cm] {};
	
	\path
	(z-j) edge [post] (w-j)
	(alpha-hyper) edge [post] (author_dist)
	(alpha-hyper) edge [post] (category_dist)
	(category_dist) edge [post] (z-j)
	(author_obs) edge [post] (z-j)
	(author_dist) edge [post] (z-j)
	(beta-hyper) edge [post] (w-j)
	(eta-hyper) edge [post] (beta-hyper)
	;
	
	\node [draw,fit=(w-j) (author_obs), inner sep=14pt] (plate-context) {};
	\node [below right] at (plate-context.north west) {$D$};
	
	\node [draw,fit=(w-j) (z-j), inner sep=10pt] (plate-token) {};
	\node [below left] at (plate-token.north east) {$N$};
	
	\node [draw,fit=(beta-hyper) (beta-hyper), inner sep=17pt] (plate-context) {};
	\node [above right] at (plate-context.south west) {$K$};
	
	\node [draw,fit=(author_dist) (author_dist), inner sep=17pt] (plate-context) {};
	\node [above right] at (plate-context.south west) {$A$};
	
	\node [draw,fit=(category_dist) (category_dist), inner sep=17pt] (plate-context) {};
	\node [above right] at (plate-context.south west) {$C$};
	
\end{tikzpicture}
		}
		\caption{Author-Category \gls{lda}.}
		\label{fig:author_category_lda}
\end{subfigure}
	\caption{Plate notation for the metadata \gls{lda} models.}
	\label{fig:metadata_lda}
\end{figure*}

\subsection{Pachinko Allocation}
In order to handle the hierarchical structure of the taxonomy metadata field, we use a hierarchical topic model, namely the \acrfull{pam} from \citet{li2006pachinko}.
Pachinko allocation generalizes \gls{lda}, making it possible to construct topic hierarchies based on any \gls{dag} structure.
\gls{pam} is a topic model focusing on finding topics of different abstraction levels and modeling the connections between these topics. 

Each node in the \gls{dag} structure represents a topic in the pachinko allocation model. 
However, unlike \gls{lda} where topics are distributions over words, in \gls{pam} topics are multinomial distributions over words and/or other topics further down the hierarchy of the \gls{dag} structure.
An example of the \gls{dag} structure used in this paper is shown in \autoref{fig:pachinko_dag}.

In this paper, we use a layered \gls{pam} meaning that the \gls{dag} structure is divided into layers, with every node in each layer fully connected to every node in the next layer of the hierarchy.
The first layer consists of only the root node, a topic which all documents are part of, and the bottom layer consists of leaf nodes, which are the only nodes to contain distributions over words, and they have no other topics in their distribution.

We replicate some layers in our \gls{dag} structure based on the structure from the taxonomy field within our dataset, having some nodes represent a topic based on a specific taxonomy.
To make the algorithm construct the topics to be based on our taxonomies, we introduce a novel locking mechanism for the Gibbs sampler which we use to run \gls{pam}.
This mechanism is discussed further in \autoref{subsec:mod_pachinko}.

We use a five-level pachinko tree structure, following the format presented by \citet{li2006pachinko}.
The first layer is the root layer.
The last layer is the word layer consisting of one node for each word in the vocabulary of our corpus.
The second and third layers will be constructed based on the entries of the first two positions in our taxonomy metadata, meaning there will be one node for each unique sub-taxonomy entry that is in the first or second position in the taxonomy sequence (e.g. "STEDER" and "Danmark", which is taken from "STEDER/Danmark/Aalborg", but not "Aalborg" since it is in the third position).
We only use the first two layers for this, as these are among the most informative, and because introducing even more layers would slow down the training exponentially, since the probability of all possible combinations of topics needs to be sampled for every word during training.
The fourth layer consists of $K = 90$ 'blank' topics, as with the other models we present in this paper.
This layer is added so that the model can construct its own topics based on the higher-level topics learned from our taxonomy metadata.
This \gls{dag} structure is visualized in \autoref{fig:pachinko_dag}.

The generative process for \gls{pam} is described as follows:
First, the multinomial distributions for each node in the \gls{dag} are sampled from Dirichlet distributions based on $\alpha$ values.
While it is possible to use different $\alpha$ values for each layer, we found through experimentation that using the same $\alpha$ value for all layers still provided good results.
Similarly, for the second to last layer, the multinomial distribution over words is sampled from a Dirichlet distribution based on a single $\beta$ value.

After all topic distributions have been sampled, documents are created by sampling words in the following manner:
For each word, a chain of topics is sampled by sampling one topic from each layer of the structure, and finally, a word is chosen based on the last topic.
More specifically, for each topic $Z_{wl}$ starting with the root topic $Z_{w1}$, we sample a topic from the next layer $Z_{w(l+1)}$ from that node's topic distribution $\theta_l^{(d)}$ based on observed data from that document.
This generative process in also illustrated in the plate notation for our \gls{pam} in \autoref{fig:pachinko}.

We use Gibbs sampling for performing inference.
For each word, a chain of topics is sampled by calculating the probability of all combinations of topics and making a weighted sample.
The probability of each topic combination is calculated using the joint probability of the topics, as presented in \autoref{eq:pachinko_gibbs}.

\begin{equation}\label{eq:pachinko_gibbs}
	\begin{split}
		P(Z_{w2} = t_a, Z_{w3} = t_b, Z_{w4} = t_c | \textbf{D}, z_{-w}, \alpha, \beta) \propto \\
		\frac{n_{1a}^d + \alpha_{1a}}{n_1^d + \sum_{a'} \alpha_{1a'}} \times
		\frac{n_{ab}^d + \alpha_{ab}}{n_a^d + \sum_{b'} \alpha_{ab'}}  \times 
		\frac{n_{bc}^d + \alpha_{bc}}{n_{b}^d + \sum_{c'} \alpha_{bc'}} \\ \times 
		\frac{n_{cw}^d + \beta_{w}}{n_{c} + \sum_{m} \beta_{m}} 
	\end{split}
\end{equation}
As in \citet{li2006pachinko}, $Z_{w2}$, $Z_{w3}$, $Z_{w4}$ are topic assignments for the three middle layers of topics in our 5-layer Pachinko model.
The root topic is not part of this equation since all words are part of it, so the probability does not need to be calculated.
$Z_{-w}$ is the word topic assignment, for all other words except the one that is being updated.
$n_x^d$ is the number of times topic $t_x$ occurs in document $d$ according to $Z_{-w}$. 
The $n_{xy}^d$ describes how many times topic $t_y$ is sampled from its parent $t_x$ within document $d$ according to $Z_{-w}$.
$n_x$ is the number of times topic $t_x$ occurs in the corpus according to $Z_{-w}$, and $n_{xw}$ is the number of times a word $w$ is in $t_x$ according to $Z_{-w}$.

\begin{figure}
	\centering
	\resizebox{0.8\columnwidth}{!}{%
	\begin{tikzpicture}
		[
		observed/.style={minimum size=25pt,circle,draw=blue!50,fill=blue!20},
		unobserved/.style={minimum size=25pt,circle,draw},
		post/.style={->,>=stealth',semithick},
		]
		% Layer 0
		\node (top) [unobserved] at (0,0) {};
		\node (topname) [right of = top, node distance=1.5cm] {Root Layer};
		
		% Layer 1
		\node (l11) [observed] at ([shift=({245:2 cm})]top) {};
		\node (l12) [observed] at ([shift=({295:2 cm})]top) {};
		\node (l1_dots) [right of = l11, node distance=0.85cm] {\scalebox{0.75}{$\bullet\bullet\bullet$}};
		\node (1name) [right of = l12, node distance=2cm] {Taxonomy Layer 1};
		
		% Layer 2
		\node (l21) [observed] at ([shift=({245:2 cm})]l11) {};
		\node (l22) [observed] at ([shift=({295:2 cm})]l11) {};
		\node (l23) [observed] at ([shift=({295:2 cm})]l12) {};
		\node (l2_dots) [right of = l22, node distance=0.85cm] {\scalebox{0.75}{$\bullet\bullet\bullet$}};
		\node (2name) [right of = l23, node distance=2cm] {Taxonomy Layer 2};
		
		% Layer 3
		\node (l31) [unobserved] at ([shift=({245:2 cm})]l21) {};
		\node (l32) [unobserved] at ([shift=({295:2 cm})]l21) {};
		\node (l33) [unobserved] at ([shift=({295:2 cm})]l23) {};
		\node (l2_dots) [right of = l32, node distance=1.75cm] {\scalebox{0.75}{$\bullet\bullet\bullet$}};
		\node (3name) [right of = l33, node distance=1.5cm] {Topic Layer};
		
		% Layer 4
		\node (l41) [unobserved] at ([shift=({270:2 cm})]l31) {};
		\node (l42) [unobserved] at ([shift=({270:2 cm})]l32) {};
		\node (l43) [unobserved] at ([shift=({270:2 cm})]l33) {};
		\node (l2_dots) [right of = l42, node distance=1.75cm] {\scalebox{0.75}{$\bullet\bullet\bullet$}};
		\node (4name) [right of = l43, node distance=1.5cm] {Word Layer};
		
		\path
		% Layer 0
		(top) edge [post] (l11)
		(top) edge [post] (l12)
		
		% Layer 1
		(l11) edge [post] (l21)
		(l11) edge [post] (l22)
		(l11) edge [post] (l23)
		(l12) edge [post] (l21)
		(l12) edge [post] (l22)
		(l12) edge [post] (l23)
		
		% Layer 2
		(l21) edge [post] (l31)
		(l21) edge [post] (l32)
		(l21) edge [post] (l33)
		(l22) edge [post] (l31)
		(l22) edge [post] (l32)
		(l22) edge [post] (l33)
		(l23) edge [post] (l31)
		(l23) edge [post] (l32)
		(l23) edge [post] (l33)
		
		% Layer 3
		(l31) edge [post] (l41)
		(l31) edge [post] (l42)
		(l31) edge [post] (l43)
		(l32) edge [post] (l41)
		(l32) edge [post] (l42)
		(l32) edge [post] (l43)
		(l33) edge [post] (l41)
		(l33) edge [post] (l42)
		(l33) edge [post] (l43)
		;
		
		
		\node (root) [unobserved, node distance=1.85cm] at (7,0) {};
		\node (rootname) [right of = root, node distance=2.5cm] {Root};
		
		\node (first) [observed, below of=root, node distance=1.85cm] {};
		\node (firstname) [right of = first, node distance=2.5cm] {STEDER};
		
		\node (second) [observed, below of=first, node distance=1.85cm] {};
		\node (secondname) [right of = second, node distance=2.5cm] {Danmark};
		
		\node (third) [unobserved, below of=second, node distance=1.85cm] {};
		\node (thirdname) [right of = third, node distance=2.5cm] {Sports topic};
		
		\node (fourth) [unobserved, below of=third, node distance=1.85cm] {};
		\node (fourthname) [right of = fourth, node distance=2.5cm] {"Football"};
		
		
		\path
		% Layer 0
		(root) edge [post] (first)
		(first) edge [post] (second)
		(second) edge [post] (third)
		(third) edge [post] (fourth)
		;
		
\end{tikzpicture}}
\end{figure}


\begin{figure}[h]
	\centering
	\resizebox{0.8\columnwidth}{!}{%
	\begin{tikzpicture}
		[
		observed/.style={minimum size=26pt,circle,draw=blue!50,fill=blue!20},
		unobserved/.style={minimum size=26pt,circle,draw},
		post/.style={->,>=stealth',semithick},
		]
		
		\node (w-j) [observed] at (0,0) {$W$};
		\node (z-4) [unobserved, above of= w-j, node distance=2.5cm] {$Z_{w4}$};
		\node (z-3) [unobserved, left of= z-4, node distance=2cm] {$Z_{w3}$};
		\node (z-2) [unobserved, left of= z-3, node distance=1.5cm] {$Z_{w2}$};
		\node (z-1) [unobserved, left of= z-2, node distance=1.5cm] {$Z_{w1}$};
		
		\node (theta_a) [unobserved, above of= z-2, node distance=2cm] {$\theta_a^d$};
		\node (alpha_a) [unobserved, above of= theta_a , node distance=2cm] {$\alpha_a$};
		
		\node (theta_b) [unobserved, above of= z-3, node distance=2cm] {$\theta_b^d$};
		\node (alpha_b) [unobserved, above of= theta_b , node distance=2cm] {$\alpha_b$};
		
		\node (theta_c) [unobserved, above of= z-4, node distance=2cm] {$\theta_c^d$};
		\node (alpha_c) [unobserved, above of= theta_c , node distance=2cm] {$\alpha_c$};
		
		\node (theta_s2) [unobserved, left of=w-j , node distance=8cm] {$\theta_c$};
		\node (beta) [unobserved, above of=theta_s2 , node distance=2.5cm] {$\beta$};
		
		\path
		(z-4) edge [post] (w-j)
		(z-3) edge [post] (z-4)
		(z-2) edge [post] (z-3)
		(z-1) edge [post] (z-2)
		
		(theta_a) edge [post] (z-2)
		(alpha_a) edge [post] (theta_a)
		
		(theta_b) edge [post] (z-3)
		(alpha_b) edge [post] (theta_b)
		
		(theta_c) edge [post] (z-4)
		(alpha_c) edge [post] (theta_c)
		
		(theta_s2) edge [post] (w-j)
		(beta) edge [post] (theta_s2)
		;
		
		\node [draw,fit=(w-j) (theta_a) (z-1), inner sep=25pt] (plate-context) {};
		\node [above left] at (plate-context.south east) {$N$};
		
		\node [draw,fit=(w-j) (z-1), inner sep=12.5pt] (plate-token) {};
		\node [above left] at (plate-token.south east) {$|d|$};
		
		\node [draw,fit=(theta_s2), inner sep=10pt] (plate-token) {};
		\node [above left] at (plate-token.south east) {$S_3$};
		
		\node [draw,fit=(theta_c) (alpha_c), inner sep=10pt] (plate-token) {};
		\node [above left] at (plate-token.south east) {$S_2$};
		
		\node [draw,fit=(theta_b) (alpha_b), inner sep=10pt] (plate-token) {};
		\node [above left] at (plate-token.south east) {$S_1$};
		
	\end{tikzpicture}}
	\caption{Plate notation for the five-level \gls{pam}.}
	\label{fig:pachinko}
\end{figure}
\vejleder[inline]{a,b,c ~ $t_i$ in Figure 5}

\subsubsection{Modification to Pachinko Allocation}\label{subsec:mod_pachinko}
Without modification, \gls{pam} will find topics with the same structure as our taxonomy, but the topics will not actually be based on the taxonomy.
However, only $25\%$ of the documents in our dataset have an observed taxonomy.
To account for this, we lock taxonomy for the words in the observed documents to be in the corresponding topics within the pachinko \gls{dag} instead of continuously sampling them using Gibbs sampling.
This creates a constant context for the taxonomy topics, which the documents with unobserved taxonomies will be fitted around.

Some of the observed documents have multiple taxonomies.
For these documents, one of the taxonomies is chosen randomly for each word in the document.

\subsection{Combinations}\label{subsec:combinations}
\vejleder[inline]{hvorfor og formål?}
We have created a combination model to see whether using multiple metadata fields at the same time to draw topics, would affect the performance of the topic model.
The first model we have created is the Author-Category combination model.
To combine the Author and Category metadata, we multiply the topic distributions $\theta_a$ and $\theta_c$ when sampling a new word topic.
Though, before sampling the word topic, we normalize the product of the multiplication to sum to 1, which also makes it a distribution. \vejleder[inline]{mathify this}
This makes both of the learned meta distributions influence the sampled word topics.
The plate notation for the Author-Category model can be seen in \autoref{fig:author_category_lda}.
