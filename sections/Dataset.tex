\section{Dataset and Preprocessing}\label{sec:dataset}
Nordjyske is a Danish news agency that maintains multiple newspapers, radios, and other news sources throughout north Jutland, a region in Denmark.
They store their news articles in a non-public database, where each article contains multiple metadata types which describe aspects of the data, e.g., the author and publication date.
The dataset we use ranges from 2017 to 2019 and contains $248,385$ articles.

We perform basic preprocessing to make the data more applicable for topic models.
Firstly, because the dataset includes articles from multiple cities and regions, duplicates do occur in the dataset.
These duplicates are removed, so only unique articles are kept.
After this, words that appear in less than 10 articles and words that appear in more than 10$\%$ of articles are filtered out.
This is done to keep words that are used enough to find patterns in topics and to remove words that are similar to stop words.
Finally, after the words are filtered out, the empty documents are removed.
After preprocessing, the dataset contains $139,060$ articles that use a vocabulary of $69,192$ unique words.

In the following sections, we describe each of the metadata types which are analyzed.
Further details about these metadata types can be seen in Appendix \autoref{sec:appendix_meta_data}.
The inclusion of a stemming process has also been tested and is described in Appendix \autoref{sec:stemming}.

The metadata types that we are working with do have some problems that can be mitigated to a degree by preprocessing.
These problems are all related to some metadata values only being used in a few documents.
Since the metadata values are used to group documents together and find common topics and words within grouped documents, metadata values that group too few documents are not very relevant and are therefore combined or removed.

\subsection{Author}
The author metadata contains the name of the author, who has written the article.
Each article only has a single author, so we do not account for multiple authors, whereas \citet{author_topic_2012} account for multiple authors.
This metadata is fully observed within the dataset, meaning that every article has an author.
Originally there were $227$ different authors within the dataset.
After combining authors that have written less than $14$ documents (${\sim}0.001\%$ of the total document set) into a 'misc' author of size $204$, $184$ authors remain.
%and they are almost evenly distributed in the number of articles they have written.

\subsection{Category}
% what it is?
The category metadata describes a variety of different aspects.
One part of the categories contains which specific newspaper the article belong to, e.g., 'Aalborg-Newspaper'.
Another part of the categories describes the overall subject of the document, such as 'Culture' and 'Sports-newspaper'.
However, there are also nonsensical categories such as '53. Frederik', that do not seem to describe the subject of the document.
% stats
This metadata is fully observed within the dataset and there are originally $58$ different categories in the dataset.
However, while most of these categories cover a significant number of documents, some categories are only used by a few documents.
After combining all categories covering less than $140$ documents (${\sim}0.01\%$ of the total document set) into a single new 'misc' category of size $229$, $34$ categories remain.
The smallest category consists of $188$ documents, and for all categories, the median category size is $3022$.
\autoref{fig:category_box} shows an overview of the size of the categories after filtering.
The categories from the 3rd quartile and up do become much larger compared to the median, and there is an outlier category with $20,241$ documents.
These larger categories seem to be about topics that are written about often or categories that cover a specific newspaper.
All of the category labels can be seen in \autoref{tab:category_table} and the statistics in \autoref{tab:meta_prepro_stats}.
\todo[inline]{move 38 to 42 into appendix (starting with 'the smallest category'}

\subsection{Taxonomy}\label{sec:dataset_taxonomy}
The taxonomy metadata describes a hierarchical sequence of the topical or geographical subjects associated with the article.
Each taxonomy sequence consists of several taxonomy entries.
This metadata is only partially observed within the dataset, which means that ${\sim}25\%$ of the articles contain this metadata.
It is also possible for articles to contain multiple taxonomy sequences.
General patterns for taxonomy sequences include:
\begin{itemize}
	\item PLACES/Country/Region/Town
	\item TOPICS/Sub-Topic/Subsub-topic
\end{itemize}
Examples of these sequences are:
\begin{itemize}
	\item PLACES/Danmark/Nordjylland/Aalborg/Lillevorde
	\item TOPICS/Religion/Christianity
\end{itemize}
About $80\%$ of the observed sequences contain the 'PLACES' variable and $20\%$ use the 'TOPICS' variable.
There are also a few other top-level taxonomy entries; however, they are not as informative and are very rarely used.
Originally there were $1135$ different taxonomy entries; however, after removing taxonomy entries used by less than $14$ documents (${\sim}0.001\%$ of the total document set), only $355$ remain.
