\section{Dataset}
Nordjyske is a Danish news agency that maintains multiple newspapers, radios and other news sources throughout north Jutland, a region in Denmark.
They store their news articles in a non-public database, where each article contains multiple meta-data fields which describes some aspect of the data eg. author.
The dataset is from 2017 to 2019 and contains 139261 articles which uses a vocabulary of 69192 unique words.
One of the metadata fields is the Category field, which both describes where the article is supposed to be located(within a newspaper) and also which subject the article is about.
These meta-data are very interesting in that they detail the data in certain ways which might be useful in some way.

In the following section, we describe each of the meta-data fields which are analysed.

\subsection{Author}
This field mentions the author, who have written the article.
This field is fully observed within the dataset, which means that every article has an author.
There are $227$ different authors within the dataset and they are almost evenly distributed in the number of articles they have written.

\subsection{Category}
The category field describes a variety of different aspects. 
This field is fully observed within the dataset and there are $58$ different categories.
A proportion of the articles contains which specific newspaper, they belong to, eg. Aalborg-Newspaper.
Another proportion of the category fields describes the overall theme, such as Culture and Sports-newspaper.


\subsection{Taxonomy}
The taxonomy field describes a hierarchical structure of the topical or geographical subject of the articles.
This field is only partially observed within the dataset, which means that roughly $50\%$ of the articles contain this field.
We observe a general pattern when traversing this field which is:
\begin{itemize}
	\item Places/Country/Region/Town
	\item Topics/Sub-Topic/Subsub-topic
\end{itemize}
Examples of this field are:
\begin{itemize}
	\item PLACES/Danmark/Nordjylland/Aalborg/Lillevorde
	\item TOPICS/Religion/Christianity
\end{itemize}