\subsection{Author-topic model}\label{sec:discussion_author_topic}
Some interesting observations can also be made specifically in the author-topic model.
One observation that is possible, is looking at the similarity of authors.
In this model, the author-topic distribution defines the probabilities of topics being written by a specific author.
Then, just as \citet{author_topic_2012}, the similarity of authors can be found by calculating the symmetric Kullback-Leibler divergence:

\begin{equation}
	sKL(i,j) = \sum_{t=1}^{T}\left[\theta_{it}\, log \frac{\theta_{it}}{\theta_{jt}} + \theta_{jt}\, log \frac{\theta_{jt}}{\theta_{it}}\right]
\end{equation}
\noindent where $\theta_{it}$ is the probability of author $i$ having written about topic $t$, and the same for $\theta_{jt}$ with author $j$.

In the context of using these similarities to assist Nordjyske, knowing how similar authors are gives the opportunity to recommend new authors to readers, while the articles are about similar topics.
In \autoref{tab:author_similarity} the top 10 author pairs, based on this similarity measure, are shown.
A smaller KL value means the authors are more similar.
The number in parenthesis next to each author is the number of articles they have written in our dataset.

\begin{table}[h]
	\centering
	\caption{Top 10 author pairs based on the symmetric KL divergence between authors.}
	\begin{tabular}{r|c}
		Author pair & KL \\
		\midrule
		Lars Termansen (328) \& Mikkel Færgemann Viken (91) & 1.50 \\
		Morten Nis Klenø (17) \& Anne Helene Thomsen (606) & 1.72 \\
		Lars Termansen (328) \& Lars Christensen (1293) & 2.43 \\
		Esben Heine Pedersen (1689) \& Caspar Birk (71) & 2.47 \\
		Lars Christensen (1293) \& Poul Christoffersen (65) & 2.53 \\
		Lone Beck (92) \& Max Melgaard (587) & 2.74 \\
		HANNE Lindblad Jensen (27) \& Peter Tordrup Larsen (5002) & 2.94 \\
		Søren Kjær (95) \& Carl Chr. Madsen (785) & 2.98 \\
		Heidi Majgaard B. Pedersen (244) \& Lisbeth Helleskov (361) & 3.05 \\
		Lars Termansen (328) \& Morten Lind (413) & 3.16 \\
		\midrule
		Maximum & 34.51 \\
		Median & 24.20 \\
	\end{tabular}
	\label{tab:author_similarity}
\end{table}

%In general does this make sense with sampled documents?
%A few examples of authors this makes a lot of sense with
