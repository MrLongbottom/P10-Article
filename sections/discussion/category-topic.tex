\subsection{Category-topic model}\label{sec:discussion_category_topic} 
Specific observations for the category-topic model can also be made.
As with the author-topic model, the similarity between pairs of categories can be calculated.
Because topic distributions are generated for each category, category similarity can also be calculated using \autoref{eq:author_similarity} where $i$ and $j$ are categories instead of authors.
In \autoref{tab:category_similarity}, the top 10 category pairs, based on symmetric KL divergence, are shown.

The second most similar pair, 'Friii' and 'Debat', is interesting to look at since 'Friii' does not seem to have a theme in the articles written, articles with the 'Debat' (debate) category seem to mostly cover themes that can bring differing opinions and articles with interviews.
This indicates that the model does not find these deeper thematic differences in articles or that it finds other patterns that are difficult to see.

It is also interesting that the 'misc' category is seen twice in the top 10 ranking even though it is made up of many smaller categories with no connection to each other.
Though, it is not surprising that this thematically mixed category is quite similar to 'Friii' and 'Debat', which are more thematically wide categories.

It is also clear that some of the topics that the model has learned fit well with how some categories are used in the dataset.
For example, the 4th ranking pair 'Sport-avis' and 'Morsø Sport' are clearly correlated by their category names covering sports news and the similarity of the topic distributions learned for both categories indicates that the model has learned these sports topics correctly.

Finally, it is worth noting that there are no category pairs, where both categories are based on geographic locations, in the top 10 pairs.
This may indicate that each city or municipality in Denmark does have some differences in which topics are written about in general.

A random selection of categories from the dataset and the top words from their most probable topic can be seen in \autoref{tab:author_top_words}.

This knowledge about categories can support recommendation in multiple ways.
An example is, that while news sites often have the possibility to filter articles based on categories, knowing which categories are similar gives further opportunities for recommending related or similar articles.

\begin{table*}[t]
	\centering
	\caption{Top 10 words of selected topics from our taxonomy-topic model. Labels have been manually added to the topics to increase readability.}
	\label{tab:pachinko_selected_topics}
	\begin{tabular}{c | c | c}
		Topic & Label & Top 10 words \\
		\hline
		8 & politics & venstre, valg, valget, partiet, partier, parti, stemmer, mette, politik, regering \\
		9 & money & procent, viser, tal, antallet, milliarder, pct, seneste, penge, millioner, indland \\
		19 & filler & mig, maske, du, folk, synes, ting, faktisk, nogen, altid, tror \\
		21 & university & unge, uddannelse, studerende, gymnasium, elever, uddannelser, universitet, procent, uddannelsen, nordjylland \\
		41 & academic research & universitet, professor, forskere, forskning, forskerne, viser, verden, institut, procent, aarhus \\
		42 & filler & mig, min, mit, ham, aldrig, gik, lille, maske, mine, altid \\
		45 & wildlife & dyr, naturen, natur, ulve, fugle, ulven, arter, dyrene, vilde, ulv \\
		47 & church & kirke, kirken, sognepræst, præst, søndag, koret, gudstjeneste, aften, kor, organist \\
		59 & music concerts & musik, koncert, sange, spiller, koncerter, band, koncerten, festival, musikken, publikum \\
		60 & buisness & virksomheden, millioner, a, direktør, procent, medarbejdere, selskabet, overskud, ansatte, virksomhed \\
		69 & primary school & elever, unge, skole, eleverne, skolen, skoler, klasse, børn, folkeskolen, lærere \\
		74 & elder care & ældre, borgere, kommunen, millioner, penge, nordjylland, plejehjem, borgerne, kommunens, budget \\
		75 & filler & du, din, dig, dit, altsa, dine, maske, nemlig, bruge, hvordan \\
		79 & filler & mig, min, hendes, hende, rigtig, arige, altid, arbejde, mor, mine \\
		86 & sports & handbold, mors, thy, hold, kamp, sæson, kampe, kampen, point, holdet \\
	\end{tabular}
\end{table*}


\begin{table*}[t]
	\centering
	\caption{IDs of the 5 most occurring fourth layer topics for each third layer topic from the taxonomy-topic model. See \autoref{tab:pachinko_selected_topics} and Appendix \autoref{tab:pachinko_topics} for the most occurring words for each ID.}
	\label{tab:pachinko_mid_topics}
	\begin{tabular}{c | c | c | c | c | c}
		Taxonomy Name & Top 5 Topic IDs & Taxonomy Name & Top 5 Topic IDs & Taxonomy Name & Top 5 Topic IDs \\ \hline
		Danmark & 8, 42, 82, 59, 79 & Udland & 42, 79, 59, 8, 32 & Kultur & 9, 42, 79, 19, 8 \\
		Landbrug & 42, 79, 8, 9, 19 & Kriminalitet & 42, 75, 60, 8, 86 & Socialstof & 42, 9, 79, 86, 8 \\
		Arbejdsmarked & 42, 79, 59, 8, 9 & Økonomi & 79, 75, 74, 42, 9 & Sundhed & 8, 32, 42, 9, 19 \\
		Politik & 42, 75, 9, 19, 74 & Musik & 75, 42, 59, 11, 79 & Sport & 42, 75, 8, 59, 52 \\
		Bolig & 75, 42, 86, 79, 8 & Videnskab & 42, 8, 52, 79, 19 & Trafik & 42, 74, 8, 52, 32 \\
		Erhverv & 42, 8, 59, 32, 79 & Uddannelse & 42, 9, 75, 32, 74 & Energi & 42, 8, 79, 19, 86 \\
		Ulykker & 42, 75, 9, 79, 32 & Fritid & 42, 8, 75, 82, 79 & Socialt & 42, 75, 79, 59, 9 \\
		Dyr & 86, 42, 79, 52, 9 & Natur & 42, 52, 9, 32, 79 & Miljø & 8, 42, 75, 52, 59 \\
		Familie & 79, 8, 42, 59, 32 & Politi & 42, 75, 79, 8, 59 & Byggeri & 75, 42, 79, 77, 59 \\
		Etik & 79, 42, 8, 86, 74 & Religion & 42, 79, 8, 59, 32 & Kommunalvalg & 42, 8, 75, 79, 32 \\
		Nordjyske Plus & 42, 86, 9, 79, 74 & DF & 42, 8, 59, 52, 19 & & \\
	\end{tabular}
\end{table*}